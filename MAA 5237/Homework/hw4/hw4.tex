\documentclass[12pt]{exam}
\usepackage{amsmath, amsfonts, amsthm, amssymb}
\usepackage{enumerate}
\usepackage{enumitem}
\usepackage{graphicx}
\usepackage{hyperref}
\usepackage{float}
\usepackage{pgfplots}
\pgfplotsset{width=10cm,compat=1.9}

\newcommand{\myhwtype}{Homework}
\newcommand{\myhwnum}{4}
\newcommand{\myname}{Nickolas Arustamyan}

\pagestyle{headandfoot}
\firstpageheadrule
\runningheadrule
\firstpageheader{\myhwtype\; \myhwnum}{\myname}{MAA 5237}
\runningheader{}{\myname}{}
\firstpagefooter{\today}{}{\thepage\,/\,\numpages}
\runningfooter{}{}{\thepage\,/\,\numpages}

\begin{document}
\begin{questions}
\question 3.2.1 \newline
In order for \[\frac{|x^\top y|^p}{\|x\|+\|y\|}\] to be differentiable, we would first need it to be continuous at $0$. This means that \[\lim_{(x,y) \rightarrow (0,0)}\frac{|x^\top y|^P}{\|x\|+\|y\|}=0\] We can make the substitution $x = r\cdot u, y = s \cdot v$ where $u,v$ are unit vectors and $r = \|x\|, s = \|y\|$. This means that we get \[\lim_{r,s\to 0}\frac{|ru^{T}\cdot sv|^{P}}{r+s}=\frac{r^{P}s^{P}}{r+s}|u^{T}v|=0\] which is true if $\lim_{r,s \rightarrow 0}\frac{r^{P}s^{P}}{r+s} = 0$. Since we are interested in $(x,y)$ near $(0,0)$, we can let $x = \varepsilon\cdot u, y = \varepsilon \cdot v$ for $\varepsilon$ small. Thus, we get 
\begin{align*}
    \lim_{r,s \rightarrow 0}\frac{r^{P}s^{P}}{r+s} &=\lim_{\varepsilon \rightarrow 0}\frac{\varepsilon^{2P}}{2\varepsilon}\\
     &= \frac{\varepsilon^{2P-1}}{2}
\end{align*}
If $p>\frac{1}{2}$, we get that $f(x,y) \to 0$ as $\varepsilon \to 0$. Thus, the function is continuous when $p>\frac{1}{2}$. If $p\leq \frac{1}{2}$, the function doesn't approach $0$ and hence the isn't continuous at $(0,0)$. Thus, we must only analyze the differentiablity of $f(x,y)$ when $p > \frac{1}{2}$.\newline
First we will consider the partial derivatives. 
\[\begin{aligned}
&\text{The partial derivatives at }(0,0)\text{ are:} \\
&&&\frac{\partial f}{\partial x_i}(0,0)=\lim_{t\to0}\frac{f(te_i,0)-f(0,0)}t=0, \\
&&&\frac{\partial f}{\partial y_i}(0,0)=\lim_{t\to0}\frac{f(0,te_i)-f(0,0)}t=0. \\
&\text{So, the gradient }\nabla f(0,0)=0
\end{aligned}\]
Clearly, since the partials are continuous, the function is also Frechet differentiable. But we must look closer at the values of $P$. Looking at the limit, we get 
\[\lim_{(x,y)\to(0,0)}\frac{f(x,y)-f(0,0)-\nabla f(0,0)\cdot(x,y)}{\|(x,y)\|}=0\] This simplifies to \[\lim_{(x,y)\to(0,0)}\frac{f(x,y)}{\|(x,y)\|}=0\] Using the substitution we did above, we get \[\frac{f(h,k)}{\|(h,k)\|}=\frac{\epsilon^{2p-1}}{2\sqrt{2}\epsilon}|u^\top v|^p=\frac{\epsilon^{2p-2}}{2\sqrt{2}}|u^\top v|^p\] Hence, for $p > 1$, we see that $2p-2>0$ and thus the limit is $0$. Thus, the function is differentiable. If $p =1$, we get $2p-2=0$ so the limit won't be $0$ and hence the function won't be differentiable. So it is only Frechet differentiable if $p > 1$. 
\question 3.2.2\newline
Let $f$ be Lipschitz and $f(x_0) = 0$. Now let $g(x) = (f(x))^2$. Consider the limit \[\lim_{x \to x_0} \frac{\|g(x)-g(x_0)- g'(x_0)(x-x_0)\|}{\|x-x_0\|}\] This can be simplified to \[\lim_{x \to x_0} \frac{\|g(x)- g'(x_0)(x-x_0)\|}{\|x-x_0\|}\] Now, looking at the derivative of $g$, we see that $\nabla g(x) = 2\cdot f(x) \cdot \nabla f(x)$ and hence $\nabla g(x_0) =  2\cdot f(x_0) \cdot \nabla f(x_0) = 0$. Thus, we are just left with \[\lim_{x \to x_0} \frac{\|(f(x))^2\|}{\|x-x_0\|}\] We can rewrite this as 
\[
    \lim_{x \to x_0} \frac{\|(f(x))^2\|}{\|x-x_0\|} = \lim_{h \to 0} \frac{\|(f(x_0+h))^2\|}{\|h\|}
\]
By the Lipschitz condition, we get that $|f(x_0+h)-f(x_0)| =|f(x_0+h)| \leq L\|h\|$. Thus, $f(x_0+h)^2 \leq L^2\|h\|^2$. Hence, the limit simplifies to 
\begin{align*}
    \lim_{h \to 0} \frac{\|(f(x_0+h))^2\|}{\|h\|} &\leq \lim_{h \to 0} \frac{L^2\|h\|^2}{\|h\|}\\
    &=\lim_{h \to 0} L^2\|h\|\\
    &= 0
\end{align*}
Thus, the function is Frechet differentiable. 
\question 3.2.3 \newline
Define $g(\lambda) = f(\lambda x + (1-\lambda)y)$. Thus, $g(1) = f(x)$ and $g(0) = f(y)$. We can compute $g'(c)$ and get $\nabla f(x)(y-x)$. Now, since $f$ is convex, so is $g$. This means we can use the tangent line inequality because $g$ is convex and see that $g(1) \geq g(0) + g'(0)(1-0)$. The tangent line approximation of $f(y)$ at $x$ is $T = f(x)+\nabla f(x)(y-x)$. But we know that $f$ is convex so that means the function is above its tangent lines. This means that $f(y) \geq f(x) + \nabla f(x) (y-x)$
\question 3.2.4 \newline
Simple computation shows that $\nabla f(0,0) = (1,0)^T$ since $f_x(0,0) = 1, f_y(0,0)=0$. Thus, the linear map would be just $f'(0,0)((x,y)-(0,0)) = x$. To evaluate Frechet differentiablity, we need to check the limit \[\lim_{(x,y)\to(0,0)}\frac{f(x,y)-f(0,0)-x}{\sqrt{x^2+y^2}}\] But we can simplify this to be $x\left(\frac{\sin(y)}{y}-1\right)$ when $y \neq 0$. But we can approximate $\frac{\sin(y)}{y}-1$ by $\frac{y^2}{6}$ using the taylor expansion. Thus, we approximate $x\left(\frac{\sin(y)}{y}-1\right)$ by $\frac{xy^2}{6}$. Hence, the expression we take the limit of becomes \[\frac{\|xy^2\|}{6\sqrt{x^2+y^2}}\] Taking the limit as $(x,y) \to (0,0)$, we get $0$ regardless of the path. Hence, the function is Frechet differentiable. 
\question 3.3.1
\question 3.3.2
\question 3.3.3
\question 3.4.1
\question 3.4.2
\question 3.4.3
\question 3.4.4
\question 3.4.6
\question 3.4.7
\question 3.4.9
\question 3.4.10

\end{questions}
\end{document}