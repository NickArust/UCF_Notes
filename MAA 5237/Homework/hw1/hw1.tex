\documentclass[12pt]{exam}
\usepackage{amsmath, amsfonts, amsthm, amssymb,amsopn}
\usepackage{enumerate}
\usepackage{enumitem}
\usepackage{graphicx}
\usepackage{hyperref}
\usepackage{float}
\usepackage{pgfplots}
\pgfplotsset{width=10cm,compat=1.9}

\newcommand{\myhwtype}{Homework}
\newcommand{\myhwnum}{1}
\newcommand{\myname}{Nickolas Arustamyan}

\pagestyle{headandfoot}
\firstpageheadrule
\runningheadrule
\firstpageheader{\myhwtype\; \myhwnum}{\myname}{MAA 5237}
\runningheader{}{\myname}{}
\firstpagefooter{\today}{}{\thepage\,/\,\numpages}
\runningfooter{}{}{\thepage\,/\,\numpages}

\begin{document}
\begin{questions}
\question Question 1.1.4\newline
In order to check if $d(\cdot, \cdot)$ is a metric, we need to check that for $x,y \in \mathbb{R}^m$ \begin{enumerate}
    \item $d(x,y) > 0$ if $x \neq y$ 
    \item $d(x,x) = 0$
    \item $d(x,y) = d(y,x)$
    \item $d(x,y) \leq d(x,z) + d(z,y)$ for any $z\in \mathbb{R}^m$
\end{enumerate}
If $x\neq y$, then there must be at least one $i \in \{1,...,m\}$ such that $x_i \neq y_i$. This means that $d(x,y) \geq 1 > 0$. Hence, the first condition is satisfied. Similarly, since $x = x$, there is no index where $x_i \neq x_i$ and hence $d(x,x) = 0$. Thus, the second condition is satisfied. Looking at the definition of $d(\cdot, \cdot)$, we see that \[ d(x,y) = |\{k:x_k \neq y_k, k = 1,...,m\}| = |\{k:y_k \neq x_k, k = 1,...,m\}| = d(y,x)\] Hence the third condition is satisfied. If $z\neq x$ or $z\neq y$, then clearly $d(x,y) \leq d(x,z) + d(z,y)$. If $z=y$ or $z=x$, we would get equality. Now, let us define sets $A, B, C$ as follows:
\[A = \{k : x_k \neq y_k\}\]
\[B = \{k : x_k \neq z_k\}\]
\[C = \{k : z_k \neq y_k\}\]
For any $k\in A$, this means that $x_k \neq y_k$. This means that either $x_k = z_k$ but $z_k \neq y_k$ OR $x_k \neq z_k$. The first case implies that $k \in C$ and the second implies that $k\in B$. This means that any index $k\in A$ must also be in either $B$ or $C$. Thus, $A \subseteq B\cup C$. Hence, $d(x,y) = |A|\leq |B\cup C| \leq |B|+|C| = d(x,z) + d(z,y)$. Hence the final condition is satisfied. 

\question Question 1.1.5\newline 
In order to check if $d(\cdot, \cdot)$ is a metric, we need to check that for $x,y \in X$ \begin{enumerate}
    \item $d(x,y) > 0$ if $x \neq y$ 
    \item $d(x,x) = 0$
    \item $d(x,y) = d(y,x)$
    \item $d(x,y) \leq d(x,z) + d(z,y)$ for any $z\in X$
\end{enumerate}
Since $\hat{\rho}(x,y) = 0 \iff x = y$, then $d(x,x) = 0$. Hence the second condition is satisfied. Similarly, as $\hat{\rho}(x,y) \geq 0 $, then we know that $d(x,y) \geq 0$ as it is the maximum of two numbers greater than $0$. Hence the first conditin is met. Looking at the definition of $d(\cdot, \cdot)$, we see that \[d(x,y) = max\{\hat{\rho}(x,y), \hat{\rho}(y,x)\} = max\{\hat{\rho}(y,x), \hat{\rho}(x,y)\} = d(y,x)\] Hence the third condition is satisfied. Finally, we must use the fact that $\hat{\rho}(x,y) \leq \hat{\rho}(x,z)+ \hat{\rho}(z,y)$ and $\hat{\rho}(y,x) \leq \hat{\rho}(y,z)+ \hat{\rho}(z,x)$ and state that 
\begin{align*}
    d(x,y) &= \max\{\hat{\rho}(x,y), \hat{\rho}(y,x)\} \\
    &\leq \max\{\hat{\rho}(x,z) + \hat{\rho}(z,y), \hat{\rho}(y,z) + \hat{\rho}(z,x)\} \\
    &\leq \max\{\hat{\rho}(x,z), \hat{\rho}(z,x)\} + \max\{\hat{\rho}(y,z), \hat{\rho}(z,y)\} \\
    &= d(x,z) + d(z,y)
    \end{align*}
    Hence, the final condition is satisfied. 

\question Question 1.1.16\newline
In order to check if $\bar{d}(\cdot, \cdot)$ is a metric, we must once again check that for $x,y \in X$ \begin{enumerate}
    \item $\bar{d}(x,y) > 0$ if $x \neq y$ 
    \item $\bar{d}(x,x) = 0$
    \item $\bar{d}(x,y) = d(y,x)$
    \item $\bar{d}(x,y) \leq d(x,z) + d(z,y)$ for any $z\in X$
\end{enumerate}
Since $d$ is already known to be a metric, then we know that $d$ satisfies that $d(f(x), f(y))> 0$ when $f(x) \neq f(y)$. Since we know that $f$ is one to one, we know that if $x \neq y$, then $f(x) \neq f(y)$ and hence $\bar{d}(x,y) > 0$ when $x\neq y$. Clearly $\bar{d}(x,x) = d(f(x),f(x)) = 0$ so the second condition is satisfied. Similarly,   $\bar{d}(x,y) = d(f(x),f(y)) = d(f(y),f(x)) = \bar{d}(y,x)$ and so the third condition is satisfied. Now, let $z\in X$. Then $\bar{d}(x,y) = d(f(x),f(y)) \leq d(f(x),f(z)) + d(f(z),f(y)) = \bar{d}(x,z) +\bar{d}(z,y)$ since we know that $d$ must satisfy the triangle inequality. Thus all conditions are met and $\bar{d}$ is a metric. 
\question Question 1.2.5\newline
Since $x_n < y_n$ for all $n\geq 1$, we know that $\sup_{n > k}(x_n) \leq \sup_{n > k}(y_n)$. Hence, by definition, $\limsup(x_n) \leq \limsup(y_n)$. Similarly,  $\inf_{n > k}(x_n) \leq \inf_{n > k}(y_n)$. Hence, by definition, $\liminf(x_n) \leq \limsup(y_n)$. \newline  An example of the first equality holding is letting $x_n = 1-\frac{2}{n}$ and $y_n = 1-\frac{1}{n}$. Then $y_n > x_n$ but $\limsup(x_n) = \limsup(y_n) = 1$. Similarly, for an example of the second equality holding is $x_n =\frac{1}{n}$ and $y_n = \frac{2}{n}$. Then $y_n > x_n$ but $\liminf(x_n) = \liminf(y_n) = 0$

\question Question 1.2.6 \newline
First we must note that $x_n \leq \sup(x_n)$ and $y_n\leq \sup(y_n)$. This means that $x_n + y_n \leq \sup(x_n)+\sup(y_n)$ and hence $\sup(x_n + y_n) \leq \sup(x_n)+\sup(y_n)$. Similarly,  $\inf(x_n + y_n) \geq \inf(x_n)+\inf(y_n)$. Now, taking the limit of both sides of $\sup(x_n + y_n) \leq \sup(x_n)+\sup(y_n)$, we get \[\lim_{k\rightarrow \infty}\sup_{n\geq k}(x_n + y_n) \leq \lim_{k\rightarrow \infty}\sup_{n\geq k}(x_n)+\lim_{k\rightarrow \infty}\sup_{n\geq k}(y_n)\] which by definition gives us \[\limsup(x_n + y_n) \leq \limsup(x_n)+\limsup(y_n)\] Taking the limit of both sides for the infimum inequality gets us \[\liminf(x_n + y_n) \leq \liminf(x_n)+\liminf(y_n)\] Now An example of the first strict inequality holding is letting $x_n = (-1)^n$ and $y_n = (-1)^{n+1}$. Then $\limsup(x_n+y_n) = 0$ but $\limsup(x_n) = \limsup(y_n) = 1$ so $\limsup(x_n) +\limsup(y_n) = 2$. An example of the second strict inequality holding is letting $x_n$ and $y_n$ be defined as above. Then $\liminf(x_n+y_n) = 0$ but $\liminf(x_n) = \liminf(y_n) = -1$ so $\limsup(x_n) +\limsup(y_n) = -2$.

\question Question 1.2.7\newline
Since $x_n \in c$, we know that $x_n$ is cauchy and since $\mathbb{R}$ is complete, it must be convergent. As it is convergent, $\limsup(x_n) = \lim_{n\rightarrow \infty}x_n = L$.  Now, we know from the previous problem that \[\limsup(x_n+y_n) \leq \limsup(x_n) + \limsup(y_n) =\lim_{n\rightarrow \infty}x_n + \limsup(y_n) = L + \limsup(y_n)\] But since $x_n \rightarrow L$, we know that for any $\varepsilon >0$, there must be an $N$ such that for all $n > N$, $|x_n - L| < \varepsilon$. Thus, taking $n>N$, we can reasonably approximate $x_n$ as $L$ and see that $\sup_{k \geq n}(x_n+y_n) \approx \sup_{k \geq n}(L+y_n) = L + \sup_{k \geq n}(y_n) $. Taking the limit, we get \[\lim_{k\rightarrow \infty}sup_{n\geq k}(x_n+y_n) = L + \lim_{k\rightarrow \infty}sup_{n\geq k}(y_n) =  \lim_{n\rightarrow \infty}x_n + \limsup(y_n)\]
\question Question 1.2.8
\begin{parts}
    \part Using 1.2.5, we know that $\limsup(x_n) \leq \limsup(y_n)$ and $\limsup(y_n) \leq \limsup(z_n)$. Thus, \[\bar{L} = \limsup(x_n) \leq \limsup(y_n) \leq \limsup(z_n) = \bar{L}\] Hence, $\limsup(y_n) = \bar{L}$. Similarly, we know that $\liminf(x_n) \leq \liminf(y_n)$ and $\liminf(y_n) \leq \liminf(z_n)$. Thus, \[\underbar{L} = \liminf(x_n) \leq \liminf(y_n) \leq \liminf(z_n) = \underbar{L}\] Hence, $\liminf(y_n) = \underbar{L}$.
    \part The Squeeze Theorem states that if $x_n \leq y_n \leq z_n$ and $x_n \rightarrow a, z_n \rightarrow a$, then $y_n \rightarrow a$. To prove it, we must first note that in order for $\lim x_n = a$, then $\limsup x_n = \liminf x_n = a$. Applying part a, we see that $\limsup y_n = a$ and $\liminf y_n = a$. Thus, $\lim y_n = a$. 
\end{parts}

\question Question 1.2.9
\begin{parts}
    \part We first note that $x_n < y_n$ for all $n$ since $a < b$. This means that $x_{n+1} = \sqrt{x_ny_n} \leq x_n$ and hence non-decreasing. Similarly, $y_{n+1} = \frac{x_n+y_n}{2} \geq y_n$ and hence non-increasing. Since $0 < a \leq x_n \leq y_n \leq b$, we know that both sequences are bounded and monotonic. Thus, each sequence is convergent by the monotone convergence theorem. Say $x_n \rightarrow L_x$ and $y_n \rightarrow L_y$. Taking the limit of the first formula, we get \[\lim_{n\rightarrow \infty}x_{n+1} = \sqrt{\lim_{n\rightarrow \infty}x_n\cdot \lim_{n\rightarrow \infty}y_n} \implies L_x = \sqrt{L_xL_y} \implies L_x^2 = L_xL_y \implies L_x = L_y\] We can safely divide by $L_x$ since $L_x > 0$. Similarly, taking the limit of the second formula and substituting in $L_x = L_y$, we get \[\lim_{n\rightarrow \infty}y_{n+1} = \frac{\lim_{n\rightarrow \infty}x_n+\lim_{n\rightarrow \infty}y_n}{2} \implies L_y = \frac{L_x+L_y}{2} = L_x\]
\end{parts}
\question Question 1.2.10\newline
Since $x_n$ converges to $a$, we know that for any $\varepsilon$, there must be an $N \in \mathbb{N}$ such that $d(x_n,a) < \varepsilon$ for all $n > N$. We can split the sum up as $S_n = \frac{1}{n}((x_1 + ... + x_N) + (x_{N+1}+...+x_n))$.Written another way, we get $S_n = \frac{1}{n} \sum_{k=1}^{N} x_k + \frac{1}{n} \sum_{k=N+1}^{n}x_k$. Now, we examine $d(S_n, a)$. 

\begin{align*}
    |S_n - a| &= \left| \frac{1}{n} \sum_{k=1}^{N} x_k + \frac{1}{n} \sum_{k=N+1}^{n} x_k - a \right| \\
              &\leq \left| \frac{1}{n} \sum_{k=1}^{N} x_k \right| + \left| \frac{1}{n} \sum_{k=N+1}^{n} x_k - a \right| \\
              &\leq \left| \frac{1}{n} \sum_{k=1}^{N} x_k \right| + \left| \frac{1}{n} (n - N) \varepsilon \right|
\end{align*}
Now we can take the limit as $n\rightarrow \infty$. Since $N$ is fixed, the first sum will tend to $0$ as $n\rightarrow\infty$. The coefficent in the second term would tend to 1 so we get $\lim_{n\rightarrow \infty} |S_n-a| \leq \varepsilon$ and so $\lim_{n\rightarrow \infty} S_n = a$.
\question Question 1.2.11 \newline
First, notice that $\frac{x_n}{1+x_n} < 1$ for all $n$ and so $x_n < 2$. Thus it is bounded. To show the sequence is increasing, we must show that $x_{n+1} - x_n = \frac{(1+x_{n-1})x_n-(1+x_n)x_{n-1}}{(1+x_{n-1})(1+x_n)} > 0$. We will do so by induction. To verify the base case, let $n=0$. Then $x_{0+1} - x_0 = x_1 - x_0 = (1+\frac{1}{1+1})-1 = 0.5 >0$. Now, assume that for some positive integer $k$, the inequality holds and $x_{k} - x_{k-1}$. This means that \[x_k-x_{k-1}= \frac{(1+x_{k-2})x_{k-1}-(1+x_{k-1})x_{k-2}}{(1+x_{k-2})(1+x_{k-1})} > 0\]. Now, examining $x_{k+1} - x_k$, we get \[\frac{x_k}{1+x_k} - \frac{x_{k-1}}{1+x_{k-1}} = \frac{(1+x_{k-1})x_k-(1+x_k)x_{k-1}}{(1+x_{k-1})(1+x_k)} = \frac{x_k - x_{k-1}}{(1+x_{k-1})(1+x_k)}\] Since the bottom is always greater than $0$ and the top we assumed to be greater than $0$, we get a positive ratio and hence $x_k - x_{k-1} > 0 \implies x_{k+1}-x_k>0$ and thus by induction, we get that $x_n$ is increasing. Thus, by the monotone convergence theorem, we know that $x_n$ converges, say to $L$. Taking the limit, we get $\lim_{n\rightarrow \infty} x_{n+1} = 1+\frac{\lim_{n\rightarrow \infty}x_n}{\lim_{n\rightarrow \infty}x_n + 1} \implies L = 1 + \frac{L}{1+L}$. This can be simplified and we get $L^2 - L - 1=0$. Solving this quadratic, we get two roots, $\frac{1+\sqrt{5}}{2}$ and $\frac{1-\sqrt{5}}{2}$. Since $L$ must be positive, we know that $L = \frac{1+\sqrt{5}}{2}$



\end{questions}
\end{document}