\documentclass[12pt]{exam}
\usepackage{amsmath, amsfonts, amsthm, amssymb}
\usepackage{enumerate}
\usepackage{enumitem}
\usepackage{graphicx}
\usepackage{hyperref}
\usepackage{float}
\usepackage{pgfplots}
\pgfplotsset{width=10cm,compat=1.9}

\newcommand{\myhwtype}{Homework}
\newcommand{\myhwnum}{3}
\newcommand{\myname}{Nickolas Arustamyan}

\pagestyle{headandfoot}
\firstpageheadrule
\runningheadrule
\firstpageheader{\myhwtype\; \myhwnum}{\myname}{MAA 5237}
\runningheader{}{\myname}{}
\firstpagefooter{\today}{}{\thepage\,/\,\numpages}
\runningfooter{}{}{\thepage\,/\,\numpages}

\begin{document}
\begin{questions}
\question Question 2.1.2
\begin{parts}
    \part Let $(X, d_X)$ be discrete and $f: X \rightarrow Y$ be a function from $X$ to $Y$, a metric space. Let $V \subseteq Y$ be open. Then $f^{-1}(V) \subseteq X$ clearly. But we know that all subsets of the discrete space is open. Thus, the preimage of open sets in $Y$ are open in $X$. Hence, the $f$ is continuous.
    \part Let $f: X \rightarrow Y$ be continuous. Then we know that \textbf{NEED TO FINISH}
\end{parts}
\question Question 2.1.3\newline
Let $f: \mathbb{R}^m \rightarrow \mathbb{R}^l$ be continuous. Then we know that for all $\varepsilon >0$ there exists a $\delta$ such that for $a,b \in \mathbb{R}^m$, if $d(a,b) < \delta$ then $d(f(a), f(b)) < \varepsilon$. Assuming the Euclidean norms for each metric space, then $d(f(a), f(b)) = \sqrt{\sum_{i = 1}^{l} (f_i(a) - f_i(b))^2}$. Since $d(f(a), f(b)) < \varepsilon$ we know that $\sqrt{\sum_{i = 1}^{l} (f_i(a) - f_i(b))^2} < \varepsilon$. But through some algebraic manipulation, we see that $\sqrt{(f_i(a)-f_i(b))^2} < \sqrt{\sum_{i = 1}^{l} (f_i(a) - f_i(b))^2} < \varepsilon$. Thus, for all $a,b$ such that $d(a,b)< \delta$, we get that $d(f_i(a), f_i(b)) < \varepsilon$. Thus, if $f$ is continuous, then so must be the component functions. Now, instead assume that all the $f_i$ are continuous. This means that for all $a,b \in \mathbb{R}^m$, we see that if $d(a,b) < \delta$, we get that $d(f_i(a), f_i(b)) < \varepsilon$ for all $i$. Hence $\sqrt{\sum_{i = 1}^{l} (f_i(a) - f_i(b))^2} < \sqrt{l \cdot \varepsilon}$ and thus function as a whole must be continous. 
\question Question 2.1.4\newline
Let $(X, d)$ be a metric space and $\varepsilon > 0$ be given. Now, suppose that $(a,b) \in X \times X$ and $(a', b') \in X \times X$ such that $\max{d(a, a'), d(b, b')} < 0.5\varepsilon = \delta$. Now we can see that $d(d(a,b), d(a',b')) \leq d(d(a,b), d(a',b)) + d(d(a', b), d(a', b')) \leq d(a, a') + d(b, b') \leq \varepsilon$. Thus, the metric is continuous. 
\question Question 2.1.6 \newline
\textbf{WHAT IS A DOMAIN???}
\question Question 2.1.7\newline
Let $G \subseteq \mathbb{R}^m$ be the topologist sin curve. That is, $G = \{0\} \times [-1,1] \cup \{(x, \sin(\frac{1}{x}): x\in [0, 1])\}$. We know from class that this is a connected set. But let $a = (0,0)$ and $b$ be any other point in $G$. Then there is no continuous function from $a$ to $b$ since we know that the topologist sin curve is not connected. Hence this is a connected but not path connected set. 
\question Question 2.1.9\newline
Let $F$ be defined as in the question and let $a\in \mathbb{R}$ be arbitrary. Then there is a specific $\lambda_0$ such that $F(a) - \varepsilon < f_{\lambda_0}(a)$ for any $\varepsilon > 0$ since $F$ is defined as the supremum. Since $f_{\lambda_0}$ is continuous, we know there is a neighborhood $G$ around $a$ such that for all $x \in G$, we get $f_{\lambda_0}(x) - \varepsilon < f_{\lambda_0}(a)$ and thus \[F(a) -\varepsilon< f_{\lambda_0}(a) - \varepsilon < f_{\lambda_0}(a) < F(a)\] and hence $F$ is lower semi-continuous.
\question Question 2.2.1 \newline
Since $f$ is continuous, we know that for any $x$ in $X$ and for any $\varepsilon > 0$, there exists a $\delta$ such that $f(a) \in B(x, \varepsilon)$ if $a \in B(x, \delta)$. This means that if we swap the roles of $\varepsilon$ and $\delta$, for the inverse function $f^{-1}: Y \rightarrow X$ we get that $f^{-1}$ is continuous. If $X$ is not compact, we aren't guarenteed that the inverse is continuous. For example, let $X = (0,1)$ and $f = \frac{2x-1}{x-x^2}$. The inverse wouldn't be continuous.
\question Question 2.2.2\newline
Since $X$ is compact, it must be complete. This means we can use the Banach fixed point theorem and state that $f$ has a fixed point. Assume that $a,b$ are both fixed points of $f$. This means that $d(f(a) , f(b)) = d(a,b) < d(a,b)$. But this is a contradiction unless $a = b$. Hence, the fixed point is also unique. 
\question Question 2.2.4
\begin{parts}
    \part As $x \rightarrow 0$, it is clear that $sin(\frac{1}{x})$ oscillates rapidly between $-1$ and $1$. Thus, as $x \rightarrow 0$
\end{parts}
\question Question 2.2.7
\question Question 2.2.8
\question Question 2.2.9
\question Question 2.2.11
\question Question 2.2.12
\question Question 2.2.15
\question Question 2.2.16
\question Question 3.1.4
\question Question 3.1.5
\question Question 3.1.8
\question Question 3.1.9
\question Question 3.1.11
\end{questions}
\end{document}