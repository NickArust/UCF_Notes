\documentclass[12pt]{exam}
\usepackage{amsmath, amsfonts, amsthm, amssymb}
\usepackage{enumerate}
\usepackage{enumitem}
\usepackage{graphicx}
\usepackage{hyperref}
\usepackage{float}
\usepackage{pgfplots}
\pgfplotsset{width=10cm,compat=1.9}

\newcommand{\myhwtype}{Homework}
\newcommand{\myhwnum}{2}
\newcommand{\myname}{Nickolas Arustamyan}

\pagestyle{headandfoot}
\firstpageheadrule
\runningheadrule
\firstpageheader{\myhwtype\; \myhwnum}{\myname}{MAA 5327}
\runningheader{}{\myname}{}
\firstpagefooter{\today}{}{\thepage\,/\,\numpages}
\runningfooter{}{}{\thepage\,/\,\numpages}

\begin{document}
\begin{questions}
\question 1.3.1 \newline
Since $E$ is a finite set, we can list its contents and say $E = \{x_1, ..., x_n\}$. Then \[E = \bigcup_{i=1}^n {x_i}\]
But each ${x_i}$ is a singleton and hence closed. Thus $E$ is the finite union of closed sets and is thus closed.

\question 1.3.2 \newline
Let $E$ be a subset of $X$. If $E$ is the empty set, it is trivially both open and closed. Assume it is non-empty. Then it must contain some element $a \in X$. If we construct a ball with radius $\frac{1}{2}$, we can see that $B_{\frac{1}{2}}(a) \subseteq E$. Thus, since we can construct an open ball around $a$ such that the ball is a subset of $E$ and $a$ was arbitrary, then $E$ is open. Now choose a set $F \subseteq X$ such that $F^c = E$. We know that $F$ must be open. But since it is open, its complement must be closed. Hence, $E$ is also closed.
\question 1.3.4\newline
By way of contradiction, assume that $B$ is a closed set such that $E \subseteq B \subseteq \bar{E}$ and $B \neq \bar{E}$. This means that $B$ contains all the elements of $E$ but not all of its limit points. But this means that there is at least one convergent sequence entirely in $B$ whose limit point is not in $B$. But this is a contradiction to the assumption that $B$ is closed. Hence, $B = \bar{E}$. 
\question 1.3.5\newline
First, assume $E \subseteq X$ is closed. Then we know that $E^c$ is open. Now assume that $\bar{x}$ is a limit point of $x_n$ but $\bar{x} \not\in E$. This means that $\bar{x} \in E^c$. Since $E^c$ is open, we can find a neighborhood around $\bar{x}$ that is entirely within $E^c$. But this is a contradiction since it would have no other points from $E$ in that neighborhood, contradicting the fact that $\bar{x}$ is a limit point of $x_n$. Thus $\bar{x}$ must be in $E$ and thus, $E$ contains its limit points. Now, to prove the other direction, assume that $E$ contains all of its limit points. We will show that $E^c$ is open. Let $a\in E^c$. Since $a \not\in E$, it can't be a limit point of $E$ by our assumption. So this means we can find a neighborhood around $a$ that contains no points from $E$. This means that the ball is entirely contained within $E^c$ and hence $E^c$ is open. Thus $E$ is closed. 
\question 1.3.6 \newline
To show that if each $E_k \subseteq X_k$ is open, $E$ would be open, take an element $x = (x_1, x_2, ..., x_m) \in E$. Then we know that we can find balls $B_k = B(x_k, \varepsilon_k) \subseteq E_k$ for each $k$. Take the minimum radius, $\varepsilon = \min(\{\varepsilon_k\})$ and make a ball $B^* = B((x_1, ..., x_n), \varepsilon)$. Clearly,  $B^* \subseteq E$ since it contains elements from each $B_k$ and hence $B^*$ is in $E$. Thus $E$ is open. \newline
To show if each $E_k \subseteq X_k$ is closed, then so must $E$ be, we will look at limit points. Since each $E_k$ is closed, they contain all of their limit points. Let $x_n = (x_1^n, x_2^n, ..., x^n_m) \in X$ be a sequence in $E_1 \times E_2 \times ... \times E_m$ that converges to $x = (x_1, x_2, ..., x_m) \in X$. We want to show that $x \in E$. Since the sequence converges, it converges in the product metric. Thus $d(x_n, x) = \sum_{k=1}^{m} d_k(x^n_k, x_k) \rightarrow 0$ as $n\rightarrow \infty$. Since the sum goes to $0$, so must each individual term, since all the terms are positive. Thus, each $x^n_k \rightarrow x_k$. But since $x_k^n \in E_k$ and $E_k$ is closed, we know that $x_k \in E_k$. Thus $x \in E$ and $E$ is closed. 
\question 1.3.7\newline
Since $G$ is open in $\mathbb{R}$, we know that for every $x \in G$, we can find an open interval $I \subset G$ such that $x \in I \subset G$. We set $I = (a_x, b_x)$ such that $a_x = \inf\{y \in \mathbb{R} : (y,x] \subset G\}$ and $b_x = \sup\{y \in \mathbb{R} : [x,y) \subset G\}$. The interval $I = (a_x, b_x)$ is by construction the largest open interval that contains $x$. These intervals are disjoint for $x_1, x_2$ parts of different connected components because otherwise, the intervals for $x_1, x_2$ would merge together. Clearly, $G$ is covered by the open intervals because each element of $G$ is contained in an intervals. To show that we can extend this to $\mathbb{R}^n$, we must use balls instead of intervals and need to show that $G = \bigcup B(x, \varepsilon_x)$. Let $x \in G \subseteq \mathbb{R}^n$. Since $G$ is open, we know that there must be some neighborhood around $x$ that is fully contained in $G$. Let the radius of that neighborhood be $\varepsilon_x$ and force it to be rational. Since we know that $\mathbb{Q}^n$ is dense, we can find a point $q_x$ such that $d(x, q_x) \leq \varepsilon_x$. Thus, we will show that $G = \bigcup B(q_x, \varepsilon_x)$. For any $x\in G$, it must be in one of the balls. Thus $G\subseteq  B(q_x, \varepsilon_x)$. Similarly, for any $x \in B(q_x, \varepsilon_x)$, it must also be in $G$. Thus, $B(q_x, \varepsilon_x)\subseteq G $ and hence, $G = \bigcup B(q_x, \varepsilon_x)$. So there are countably many disjoint open balls that combine together to equal the original open set. 
\question 1.4.6 \newline
Let $A_n$ have the qualities described in the problem and let $a_n \in A_n$. Make a sequence $b_n = {diam(A_n)}$. We know that this sequence converges to $0$. This means that for any $\varepsilon$, there exists an $N$ such that $d(b_n, 0) < \varepsilon$ for $n > N$. Let $a^1_n, a^2_n \in A_n$. Then we know that $d(a^1_n, a^2_n) \leq \varepsilon$. Even more, we know that for $n,m > N$ we get that $d(a_n, a_n) \leq \varepsilon$. This means that the sequence is Cauchy. But since $(X, d)$ is complete, we know that $a_n$ is convergent. The Nested compact sets theorem talks about compact sets but this result doesn't make any statement or assertion about the compactness or lack thereof of these sets, just that the they are bounded essentially and that the bound shrinks to $0$ as the nesting increases. But they are very similar results.
\question 1.4.7 \newline
To show that $Q \cap [0,1]$ is totally bounded, we need to show that for every $\varepsilon > 0$, there exist a finite $n$ such that $x_1, x_2, ..., x_n \in X$ and $X \subseteq \bigcup_{i=1}^{n} B(x_i, \varepsilon)$. Let $\varepsilon>0$ be given. Then we know that there must be some $p,q \in \mathbb{N}$ such that $\frac{p}{q} < \varepsilon $. Then create a set $a_n$ where $a_0 = 0$ and $a_i = n\cdot \frac{p}{2q}$ and $a_i \leq 1$. If $n+1$ is the first index where $a_{n+1} > 1$, then we remove $a_{n+1}$ from the set and stop the process. We know the process must terminate since we require that $n\cdot \frac{p}{q} < 1$. Thus, it will terminate when $ n > \frac{q}{p}$ which must happen at some point. Then by construction, $Q \cap [0,1] \subseteq \bigcup_{i=1}^{n} B(a_i, \frac{p}{q}) \subseteq \bigcup_{i=1}^{n} B(a_i, \varepsilon)$ and so the set is totally bounded.
\question 1.4.8\newline
\begin{parts}
    \part In order to show that $\ell^p$ is seperable, we need to show that it contains a countable and dense subset. Let $x \in l^p$. This means that $x = (x_1, x_2, ..., x_n, x_{n+1}, ...)$ such that $(\sum x_i^p)^{\frac{1}{p}}$ is finite. Since the series is finite, it converges to some value and thus, the terms tend to $0$. Thus, for any $\varepsilon>0$, we can find an $N$ such that $|x_n|<\varepsilon$ for $n>N$. So we can approximate $x$ by $y_n = (x_1, x_2, ..., x_n, 0, 0, ...)$. Taking the difference between them and raising to the power $p$, we get $|x-y|_p^p = |x_{n+1}|^p + |x_{n+2}|^p+|x_{n+3}|^p + ... $. Now, we take $q = \{q_1, ..., q_n, 0, 0, ...\} \in \mathbb{Q}$ such that  $|x1-q_1|_p^p < \varepsilon,  |x2-q_2|_p^p < \frac{\varepsilon}{2}, |x2-q_2|_p^p < \frac{\varepsilon}{2^2}, ..., |x2-q_2|_p^p < \frac{\varepsilon}{2^n}$. Then $|y_n-q|_p = (\sum_{i=1}^{n} |xi-q_i|_p^p)^\frac{1}{p} < 2\varepsilon $. Now, we can use the triangle inequality and see that $|x-q| \leq |x-y| + |y-q| \leq  2\varepsilon + 2\varepsilon = 4\varepsilon$. Thus, $\mathbb{Q}$ is dense in $\ell^p$ and we already know it is countable. Thus, $\ell^p$ is seperable. 
    \part Assume by way of contradiction that $\ell^\infty$ is seperable. That means it contains a dense and countable subset, call it $D$. Now, define $B = \{[a_i]: a_i \in \{0,1\}\}$ be the set of all sequences whose terms are either $0$ or $1$. This means that for $a,b\in B$ distinct, $|a-b|_\infty = 1$ since they must differ in at least one index and thus, the distance between them is $1$. But since $D$ is dense in $\ell^\infty$, we can find some sequence $d_n \in D$ such that $d_n \in B(a, \frac{1}{3})$. But we know that these open balls are disjoint for all $a \in B$. Since $D$ is countable, we can find an injection from $A$ to $\mathbb{N}$ by $a \mapsto n(a)$ by associating $a$ with its approximation from $D$. But this means that $A$ is countable. However, due to Cantors argument, we know it cannot be countable. Thus, we have found a contradiction and thus, $\ell^\infty$ is non seperable. 
\end{parts}
\question 1.4.10 
\begin{parts}
    \part Let $K_1, ..., K_n$ be compact. Then there exist finite open covers $G_i$ for each $K_i$. Taking the union of the covers, this must be another open, finite set since it was the finite union of finite sets. It is clear that it covers $K = \bigcup_{i=1}^n K_i$ since for any element $x \in K$, it must be in one of the $K_i$ and hence must be in $K_i$'s cover and hence in the union of covers. 
    \part Let $K = \bigcap_{\lambda \in \Lambda} K_\lambda$. Then let $x_n \in K$. This means that it must have been in all of $K_\lambda$ in the family. Since they are all compact, $x_n$ must have a convergent subsequence. But this means that any sequence in $K$ has a convergent subsequence and thus $K$ is compact. 
\end{parts}
\question 1.4.11 \newline
As far as I understand, I am only being asked to investigate what happens when we force closure and not compactness. If we don't force compactness, then we cannot guarentee that the infimum, $a$, will be positive or be attained. For example, let $K_1 = \{(x,0): x \geq 1\}$ and $K_2 = \{(x, \frac{1}{x}: x \geq 1)\}$. Both are closed but not compact. as $n \rightarrow \infty$, it is clear that the distance between the two sets goes to $0$ so $a = 0$. This is clearly not positive. In addition, there is no pair of points $\bar{x} = K_1,\bar{y} \in K_2$ such that $d(\bar{x}, \bar{y}) = 0$. 

\end{questions}
\end{document}