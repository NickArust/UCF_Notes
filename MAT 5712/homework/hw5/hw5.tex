\documentclass[12pt]{exam}
\usepackage{amsmath, amsfonts, amsthm, amssymb}
\usepackage{enumerate}
\usepackage{enumitem}
\usepackage{graphicx}
\usepackage{hyperref}
\usepackage{float}
\usepackage{pgfplots}
\pgfplotsset{width=10cm,compat=1.9}

\newcommand{\myhwtype}{Homework}
\newcommand{\myhwnum}{5}
\newcommand{\myname}{Nickolas Arustamyan}

\pagestyle{headandfoot}
\firstpageheadrule
\runningheadrule
\firstpageheader{\myhwtype\; \myhwnum}{\myname}{MAT 5712}
\runningheader{}{\myname}{}
\firstpagefooter{\today}{}{\thepage\,/\,\numpages}
\runningfooter{}{}{\thepage\,/\,\numpages}

\begin{document}
\begin{questions}
\question Question 1 \newline
The second order Legendre Polynomials are given by $P_2(x) = \frac{1}{2}(3x^2-1)$. The nodes are the $x$ such that $P_2(x) = \frac{1}{2}(3x^2-1) =0$. Solving for $x$, we get $x_{1,2} = \pm \frac{1}{\sqrt{3}}$. Since these nodes are found for the interval $[-1,1]$ but we need them for $[-2,0]$, we need to transform them to find the new nodes. They would be $x' = \frac{0-(-2)}{2}x+\frac{0+(-2)}{2}$. This would result in $x'_1 = -1-\frac{1}{\sqrt{3}}$ and $x'_2 = -1+\frac{1}{\sqrt{3}}$. The weights would be $w_1=w_2=1$ on $[-1,1]$. On $[-2,0]$, these would still be $w'_1 = w'_2=1$. The integral can now be approximated as $I = w'_1f(x'_1)+w'_2f(x'_2)$. Thus we get 
\begin{align*}
    I &= w'_1f(x'_1)+w'_2f(x'_2)\\
    &= 1*e^{-(-1-\frac{1}{\sqrt{3}})}+1*e^{-(-1+\frac{1}{\sqrt{3}})}\\
    &= e^{1+\frac{1}{\sqrt{3}}}+e^{1-\frac{1}{\sqrt{3}}}\\
\end{align*}
\question Question 2\newline
The corresponding RK Scheme would be \[\xi_1 = y_n + \frac{1}{2}hf(t_n+\frac{1}{2}h, \xi_1) \]
\[y_{n+1} = y_n + hf(t_n+\frac{1}{2}h, \xi_1)\]
To find the relationship between the two, we can multiply $\xi_1$ by $2$ and then subtract from $y_{n+1}$. This gives us 
\begin{align*}
    2\xi_1-y_{n+1} &= 2y_n+hf(t_n+\frac{1}{2}h,\xi_1)-y_n+hf(t_n+\frac{1}{2}h,\xi_1)\\
    &=y_n\\
\end{align*}
This implies $2\xi_1 = y_n+y_{n+1}$ or that $\xi_1=\frac{1}{2}(y_n+y_{n+1})$. This means that \[y_{n+1} = y_n + hf(t_n+\frac{1}{2}h, \xi_1) = y_{n+1} = y_n + hf(t_n+\frac{1}{2}h,\frac{1}{2}(y_n+y_{n+1}) )\] as intended. Thus, this gives us the same information as we had originally. 

\end{questions}
\end{document}