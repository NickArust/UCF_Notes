\documentclass[12pt]{exam}
\usepackage{amsmath, amsfonts, amsthm, amssymb}
\usepackage{enumerate}
\usepackage{enumitem}
\usepackage{graphicx}
\usepackage{hyperref}
\usepackage{float}
\usepackage{pgfplots}
\pgfplotsset{width=10cm,compat=1.9}

\newcommand{\myhwtype}{Homework}
\newcommand{\myhwnum}{2}
\newcommand{\myname}{Nickolas Arustamyan}

\pagestyle{headandfoot}
\firstpageheadrule
\runningheadrule
\firstpageheader{\myhwtype\; \myhwnum}{\myname}{MAS 5145}
\runningheader{}{\myname}{}
\firstpagefooter{\today}{}{\thepage\,/\,\numpages}
\runningfooter{}{}{\thepage\,/\,\numpages}

\begin{document}
\begin{questions}
\question Question 1
Let $\alpha\beta$ be invertible. Since $dim(V) < \infty$, we know that $\alpha\beta$ is injective and hence, $ker(\alpha\beta) = \{0\}$. For any $v \in ker(\beta)$, we know that $(\alpha\beta)(v) = \alpha(\beta(v)) = \alpha(0) = 0$ since the kernal of $\alpha\beta$ is trivial. But this means that $\beta$ is injective and hence invertible since otherwise, the kernal wouldn't be trivial. Similarly, $\alpha$ is injective and thus invertible. 

\question Question 2\newline
We can easily apply the dimension formula for subspaces and see that $dim(Im(\alpha) + Im(\beta)) = dim(Im(\alpha))+dim(Im(\beta)) - dim(Im(\alpha)\cap Im(\beta))$ and thus $rank(\alpha+\beta)$

\question Question 3
\begin{parts}
\part To show that $\alpha$ is linear, we need to show that $\alpha(f+kg) = \alpha(f)+k\alpha(g)$ for all $f,g \in V$ and $k\in \mathbb{R}$. It is easy to see that $\alpha(f+kg) = \begin{bmatrix}
    f'(0)+kg'(0) & 2(f(1)+kg(1))\\
    0 & f''(3)+kg''(3)
\end{bmatrix}  = \begin{bmatrix}
    f'(0) & 2f(1)\\
    0 & f''(3)
\end{bmatrix} + k\begin{bmatrix}
    g'(0) & 2g(1)\\
    0 & g''(3)
\end{bmatrix} =\alpha(f)+k\alpha(g)$
And thus it is linear.

\part We need $\alpha(f(x)) = \begin{bmatrix}
    0 & 0 \\
    0 & 0
\end{bmatrix}$. This means that $f'(0) = 0, 2f(1) = 0,$ and $f''(3) = 0$. Since $f \in P_2(\mathbb{R})$, we know that $f = ax^2+bx+c$ for some $a,b,c \in \mathbb{R}$. This means that $f'(x) = 2ax+b$ and $f''(x) = 2a$. Thus, in order to satisfy the last condition, $a = 0$. Hence $b = 0$ to satisfy the first condition. This means that $c=0$ to satisfy the middle one. Thus, $\alpha$ is injective since the kernal is the zero function. Hence $rank(\alpha) = 3$. 
\part We can write $B_1 = 1E_{11} + 4E_{12}, B_2 = 1E_{11} - 4E_{12}, B_3 = 2E_{11} + 2E_{12} + 2E_{22}$. 
\end{parts}

\question Question 4 \newline
It is clear that $spec(\alpha^{-1}\alpha)


\end{questions}
\end{document}