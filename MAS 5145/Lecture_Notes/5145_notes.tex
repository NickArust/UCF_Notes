\documentclass{report}

%%%%%%%%%%%%%%%%%%%%%%%%%%%%%%%%%
% PACKAGE IMPORTS
%%%%%%%%%%%%%%%%%%%%%%%%%%%%%%%%%


\usepackage[tmargin=2cm,rmargin=1in,lmargin=1in,margin=0.85in,bmargin=2cm,footskip=.2in]{geometry}
\usepackage{amsmath,amsfonts,amsthm,amssymb,mathtools}
\usepackage[varbb]{newpxmath}
\usepackage{xfrac}
\usepackage[makeroom]{cancel}
\usepackage{mathtools}
\usepackage{bookmark}
\usepackage{enumitem}
\usepackage{hyperref,theoremref}
\hypersetup{
	pdftitle={Assignment},
	colorlinks=true, linkcolor=doc!90,
	bookmarksnumbered=true,
	bookmarksopen=true
}
\usepackage[most,many,breakable]{tcolorbox}
\usepackage{xcolor}
\usepackage{varwidth}
\usepackage{varwidth}
\usepackage{etoolbox}
%\usepackage{authblk}
\usepackage{nameref}
\usepackage{multicol,array}
\usepackage{tikz-cd}
\usepackage[ruled,vlined,linesnumbered]{algorithm2e}
\usepackage{comment} % enables the use of multi-line comments (\ifx \fi) 
\usepackage{import}
\usepackage{xifthen}
\usepackage{pdfpages}
\usepackage{transparent}

\newcommand\mycommfont[1]{\footnotesize\ttfamily\textcolor{blue}{#1}}
\SetCommentSty{mycommfont}
\newcommand{\incfig}[1]{%
    \def\svgwidth{\columnwidth}
    \import{./figures/}{#1.pdf_tex}
}

\usepackage{tikzsymbols}
\renewcommand\qedsymbol{$\Laughey$}


%\usepackage{import}
%\usepackage{xifthen}
%\usepackage{pdfpages}
%\usepackage{transparent}


%%%%%%%%%%%%%%%%%%%%%%%%%%%%%%
% SELF MADE COLORS
%%%%%%%%%%%%%%%%%%%%%%%%%%%%%%



\definecolor{myg}{RGB}{56, 140, 70}
\definecolor{myb}{RGB}{45, 111, 177}
\definecolor{myr}{RGB}{199, 68, 64}
\definecolor{mytheorembg}{HTML}{F2F2F9}
\definecolor{mytheoremfr}{HTML}{00007B}
\definecolor{mylenmabg}{HTML}{FFFAF8}
\definecolor{mylenmafr}{HTML}{983b0f}
\definecolor{mypropbg}{HTML}{f2fbfc}
\definecolor{mypropfr}{HTML}{191971}
\definecolor{myexamplebg}{HTML}{F2FBF8}
\definecolor{myexamplefr}{HTML}{88D6D1}
\definecolor{myexampleti}{HTML}{2A7F7F}
\definecolor{mydefinitbg}{HTML}{E5E5FF}
\definecolor{mydefinitfr}{HTML}{3F3FA3}
\definecolor{notesgreen}{RGB}{0,162,0}
\definecolor{myp}{RGB}{197, 92, 212}
\definecolor{mygr}{HTML}{2C3338}
\definecolor{myred}{RGB}{127,0,0}
\definecolor{myyellow}{RGB}{169,121,69}
\definecolor{myexercisebg}{HTML}{F2FBF8}
\definecolor{myexercisefg}{HTML}{88D6D1}


%%%%%%%%%%%%%%%%%%%%%%%%%%%%
% TCOLORBOX SETUPS
%%%%%%%%%%%%%%%%%%%%%%%%%%%%

\setlength{\parindent}{1cm}
%================================
% THEOREM BOX
%================================

\tcbuselibrary{theorems,skins,hooks}
\newtcbtheorem[number within=section]{Theorem}{Theorem}
{%
	enhanced,
	breakable,
	colback = mytheorembg,
	frame hidden,
	boxrule = 0sp,
	borderline west = {2pt}{0pt}{mytheoremfr},
	sharp corners,
	detach title,
	before upper = \tcbtitle\par\smallskip,
	coltitle = mytheoremfr,
	fonttitle = \bfseries\sffamily,
	description font = \mdseries,
	separator sign none,
	segmentation style={solid, mytheoremfr},
}
{th}

\tcbuselibrary{theorems,skins,hooks}
\newtcbtheorem[number within=chapter]{theorem}{Theorem}
{%
	enhanced,
	breakable,
	colback = mytheorembg,
	frame hidden,
	boxrule = 0sp,
	borderline west = {2pt}{0pt}{mytheoremfr},
	sharp corners,
	detach title,
	before upper = \tcbtitle\par\smallskip,
	coltitle = mytheoremfr,
	fonttitle = \bfseries\sffamily,
	description font = \mdseries,
	separator sign none,
	segmentation style={solid, mytheoremfr},
}
{th}


\tcbuselibrary{theorems,skins,hooks}
\newtcolorbox{Theoremcon}
{%
	enhanced
	,breakable
	,colback = mytheorembg
	,frame hidden
	,boxrule = 0sp
	,borderline west = {2pt}{0pt}{mytheoremfr}
	,sharp corners
	,description font = \mdseries
	,separator sign none
}

%================================
% Corollery
%================================
\tcbuselibrary{theorems,skins,hooks}
\newtcbtheorem[number within=section]{Corollary}{Corollary}
{%
	enhanced
	,breakable
	,colback = myp!10
	,frame hidden
	,boxrule = 0sp
	,borderline west = {2pt}{0pt}{myp!85!black}
	,sharp corners
	,detach title
	,before upper = \tcbtitle\par\smallskip
	,coltitle = myp!85!black
	,fonttitle = \bfseries\sffamily
	,description font = \mdseries
	,separator sign none
	,segmentation style={solid, myp!85!black}
}
{th}
\tcbuselibrary{theorems,skins,hooks}
\newtcbtheorem[number within=chapter]{corollary}{Corollary}
{%
	enhanced
	,breakable
	,colback = myp!10
	,frame hidden
	,boxrule = 0sp
	,borderline west = {2pt}{0pt}{myp!85!black}
	,sharp corners
	,detach title
	,before upper = \tcbtitle\par\smallskip
	,coltitle = myp!85!black
	,fonttitle = \bfseries\sffamily
	,description font = \mdseries
	,separator sign none
	,segmentation style={solid, myp!85!black}
}
{th}


%================================
% LENMA
%================================

\tcbuselibrary{theorems,skins,hooks}
\newtcbtheorem[number within=section]{Lenma}{Lenma}
{%
	enhanced,
	breakable,
	colback = mylenmabg,
	frame hidden,
	boxrule = 0sp,
	borderline west = {2pt}{0pt}{mylenmafr},
	sharp corners,
	detach title,
	before upper = \tcbtitle\par\smallskip,
	coltitle = mylenmafr,
	fonttitle = \bfseries\sffamily,
	description font = \mdseries,
	separator sign none,
	segmentation style={solid, mylenmafr},
}
{th}

\tcbuselibrary{theorems,skins,hooks}
\newtcbtheorem[number within=chapter]{lenma}{Lenma}
{%
	enhanced,
	breakable,
	colback = mylenmabg,
	frame hidden,
	boxrule = 0sp,
	borderline west = {2pt}{0pt}{mylenmafr},
	sharp corners,
	detach title,
	before upper = \tcbtitle\par\smallskip,
	coltitle = mylenmafr,
	fonttitle = \bfseries\sffamily,
	description font = \mdseries,
	separator sign none,
	segmentation style={solid, mylenmafr},
}
{th}


%================================
% PROPOSITION
%================================

\tcbuselibrary{theorems,skins,hooks}
\newtcbtheorem[number within=section]{Prop}{Proposition}
{%
	enhanced,
	breakable,
	colback = mypropbg,
	frame hidden,
	boxrule = 0sp,
	borderline west = {2pt}{0pt}{mypropfr},
	sharp corners,
	detach title,
	before upper = \tcbtitle\par\smallskip,
	coltitle = mypropfr,
	fonttitle = \bfseries\sffamily,
	description font = \mdseries,
	separator sign none,
	segmentation style={solid, mypropfr},
}
{th}

\tcbuselibrary{theorems,skins,hooks}
\newtcbtheorem[number within=chapter]{prop}{Proposition}
{%
	enhanced,
	breakable,
	colback = mypropbg,
	frame hidden,
	boxrule = 0sp,
	borderline west = {2pt}{0pt}{mypropfr},
	sharp corners,
	detach title,
	before upper = \tcbtitle\par\smallskip,
	coltitle = mypropfr,
	fonttitle = \bfseries\sffamily,
	description font = \mdseries,
	separator sign none,
	segmentation style={solid, mypropfr},
}
{th}


%================================
% CLAIM
%================================

\tcbuselibrary{theorems,skins,hooks}
\newtcbtheorem[number within=section]{claim}{Claim}
{%
	enhanced
	,breakable
	,colback = myg!10
	,frame hidden
	,boxrule = 0sp
	,borderline west = {2pt}{0pt}{myg}
	,sharp corners
	,detach title
	,before upper = \tcbtitle\par\smallskip
	,coltitle = myg!85!black
	,fonttitle = \bfseries\sffamily
	,description font = \mdseries
	,separator sign none
	,segmentation style={solid, myg!85!black}
}
{th}



%================================
% Exercise
%================================

\tcbuselibrary{theorems,skins,hooks}
\newtcbtheorem[number within=section]{Exercise}{Exercise}
{%
	enhanced,
	breakable,
	colback = myexercisebg,
	frame hidden,
	boxrule = 0sp,
	borderline west = {2pt}{0pt}{myexercisefg},
	sharp corners,
	detach title,
	before upper = \tcbtitle\par\smallskip,
	coltitle = myexercisefg,
	fonttitle = \bfseries\sffamily,
	description font = \mdseries,
	separator sign none,
	segmentation style={solid, myexercisefg},
}
{th}

\tcbuselibrary{theorems,skins,hooks}
\newtcbtheorem[number within=chapter]{exercise}{Exercise}
{%
	enhanced,
	breakable,
	colback = myexercisebg,
	frame hidden,
	boxrule = 0sp,
	borderline west = {2pt}{0pt}{myexercisefg},
	sharp corners,
	detach title,
	before upper = \tcbtitle\par\smallskip,
	coltitle = myexercisefg,
	fonttitle = \bfseries\sffamily,
	description font = \mdseries,
	separator sign none,
	segmentation style={solid, myexercisefg},
}
{th}

%================================
% EXAMPLE BOX
%================================

\newtcbtheorem[number within=section]{Example}{Example}
{%
	colback = myexamplebg
	,breakable
	,colframe = myexamplefr
	,coltitle = myexampleti
	,boxrule = 1pt
	,sharp corners
	,detach title
	,before upper=\tcbtitle\par\smallskip
	,fonttitle = \bfseries
	,description font = \mdseries
	,separator sign none
	,description delimiters parenthesis
}
{ex}

\newtcbtheorem[number within=chapter]{example}{Example}
{%
	colback = myexamplebg
	,breakable
	,colframe = myexamplefr
	,coltitle = myexampleti
	,boxrule = 1pt
	,sharp corners
	,detach title
	,before upper=\tcbtitle\par\smallskip
	,fonttitle = \bfseries
	,description font = \mdseries
	,separator sign none
	,description delimiters parenthesis
}
{ex}

%================================
% DEFINITION BOX
%================================

\newtcbtheorem[number within=section]{Definition}{Definition}{enhanced,
	before skip=2mm,after skip=2mm, colback=red!5,colframe=red!80!black,boxrule=0.5mm,
	attach boxed title to top left={xshift=1cm,yshift*=1mm-\tcboxedtitleheight}, varwidth boxed title*=-3cm,
	boxed title style={frame code={
					\path[fill=tcbcolback]
					([yshift=-1mm,xshift=-1mm]frame.north west)
					arc[start angle=0,end angle=180,radius=1mm]
					([yshift=-1mm,xshift=1mm]frame.north east)
					arc[start angle=180,end angle=0,radius=1mm];
					\path[left color=tcbcolback!60!black,right color=tcbcolback!60!black,
						middle color=tcbcolback!80!black]
					([xshift=-2mm]frame.north west) -- ([xshift=2mm]frame.north east)
					[rounded corners=1mm]-- ([xshift=1mm,yshift=-1mm]frame.north east)
					-- (frame.south east) -- (frame.south west)
					-- ([xshift=-1mm,yshift=-1mm]frame.north west)
					[sharp corners]-- cycle;
				},interior engine=empty,
		},
	fonttitle=\bfseries,
	title={#2},#1}{def}
\newtcbtheorem[number within=chapter]{definition}{Definition}{enhanced,
	before skip=2mm,after skip=2mm, colback=red!5,colframe=red!80!black,boxrule=0.5mm,
	attach boxed title to top left={xshift=1cm,yshift*=1mm-\tcboxedtitleheight}, varwidth boxed title*=-3cm,
	boxed title style={frame code={
					\path[fill=tcbcolback]
					([yshift=-1mm,xshift=-1mm]frame.north west)
					arc[start angle=0,end angle=180,radius=1mm]
					([yshift=-1mm,xshift=1mm]frame.north east)
					arc[start angle=180,end angle=0,radius=1mm];
					\path[left color=tcbcolback!60!black,right color=tcbcolback!60!black,
						middle color=tcbcolback!80!black]
					([xshift=-2mm]frame.north west) -- ([xshift=2mm]frame.north east)
					[rounded corners=1mm]-- ([xshift=1mm,yshift=-1mm]frame.north east)
					-- (frame.south east) -- (frame.south west)
					-- ([xshift=-1mm,yshift=-1mm]frame.north west)
					[sharp corners]-- cycle;
				},interior engine=empty,
		},
	fonttitle=\bfseries,
	title={#2},#1}{def}



%================================
% Solution BOX
%================================

\makeatletter
\newtcbtheorem{question}{Question}{enhanced,
	breakable,
	colback=white,
	colframe=myb!80!black,
	attach boxed title to top left={yshift*=-\tcboxedtitleheight},
	fonttitle=\bfseries,
	title={#2},
	boxed title size=title,
	boxed title style={%
			sharp corners,
			rounded corners=northwest,
			colback=tcbcolframe,
			boxrule=0pt,
		},
	underlay boxed title={%
			\path[fill=tcbcolframe] (title.south west)--(title.south east)
			to[out=0, in=180] ([xshift=5mm]title.east)--
			(title.center-|frame.east)
			[rounded corners=\kvtcb@arc] |-
			(frame.north) -| cycle;
		},
	#1
}{def}
\makeatother

%================================
% SOLUTION BOX
%================================

\makeatletter
\newtcolorbox{solution}{enhanced,
	breakable,
	colback=white,
	colframe=myg!80!black,
	attach boxed title to top left={yshift*=-\tcboxedtitleheight},
	title=Solution,
	boxed title size=title,
	boxed title style={%
			sharp corners,
			rounded corners=northwest,
			colback=tcbcolframe,
			boxrule=0pt,
		},
	underlay boxed title={%
			\path[fill=tcbcolframe] (title.south west)--(title.south east)
			to[out=0, in=180] ([xshift=5mm]title.east)--
			(title.center-|frame.east)
			[rounded corners=\kvtcb@arc] |-
			(frame.north) -| cycle;
		},
}
\makeatother

%================================
% Question BOX
%================================

\makeatletter
\newtcbtheorem{qstion}{Question}{enhanced,
	breakable,
	colback=white,
	colframe=mygr,
	attach boxed title to top left={yshift*=-\tcboxedtitleheight},
	fonttitle=\bfseries,
	title={#2},
	boxed title size=title,
	boxed title style={%
			sharp corners,
			rounded corners=northwest,
			colback=tcbcolframe,
			boxrule=0pt,
		},
	underlay boxed title={%
			\path[fill=tcbcolframe] (title.south west)--(title.south east)
			to[out=0, in=180] ([xshift=5mm]title.east)--
			(title.center-|frame.east)
			[rounded corners=\kvtcb@arc] |-
			(frame.north) -| cycle;
		},
	#1
}{def}
\makeatother

\newtcbtheorem[number within=chapter]{wconc}{Wrong Concept}{
	breakable,
	enhanced,
	colback=white,
	colframe=myr,
	arc=0pt,
	outer arc=0pt,
	fonttitle=\bfseries\sffamily\large,
	colbacktitle=myr,
	attach boxed title to top left={},
	boxed title style={
			enhanced,
			skin=enhancedfirst jigsaw,
			arc=3pt,
			bottom=0pt,
			interior style={fill=myr}
		},
	#1
}{def}



%================================
% NOTE BOX
%================================

\usetikzlibrary{arrows,calc,shadows.blur}
\tcbuselibrary{skins}
\newtcolorbox{note}[1][]{%
	enhanced jigsaw,
	colback=gray!20!white,%
	colframe=gray!80!black,
	size=small,
	boxrule=1pt,
	title=\textbf{Note:-},
	halign title=flush center,
	coltitle=black,
	breakable,
	drop shadow=black!50!white,
	attach boxed title to top left={xshift=1cm,yshift=-\tcboxedtitleheight/2,yshifttext=-\tcboxedtitleheight/2},
	minipage boxed title=1.5cm,
	boxed title style={%
			colback=white,
			size=fbox,
			boxrule=1pt,
			boxsep=2pt,
			underlay={%
					\coordinate (dotA) at ($(interior.west) + (-0.5pt,0)$);
					\coordinate (dotB) at ($(interior.east) + (0.5pt,0)$);
					\begin{scope}
						\clip (interior.north west) rectangle ([xshift=3ex]interior.east);
						\filldraw [white, blur shadow={shadow opacity=60, shadow yshift=-.75ex}, rounded corners=2pt] (interior.north west) rectangle (interior.south east);
					\end{scope}
					\begin{scope}[gray!80!black]
						\fill (dotA) circle (2pt);
						\fill (dotB) circle (2pt);
					\end{scope}
				},
		},
	#1,
}

%%%%%%%%%%%%%%%%%%%%%%%%%%%%%%
% SELF MADE COMMANDS
%%%%%%%%%%%%%%%%%%%%%%%%%%%%%%


\newcommand{\thm}[2]{\begin{Theorem}{#1}{}#2\end{Theorem}}
\newcommand{\cor}[2]{\begin{Corollary}{#1}{}#2\end{Corollary}}
\newcommand{\mlenma}[2]{\begin{Lenma}{#1}{}#2\end{Lenma}}
\newcommand{\mprop}[2]{\begin{Prop}{#1}{}#2\end{Prop}}
\newcommand{\clm}[3]{\begin{claim}{#1}{#2}#3\end{claim}}
\newcommand{\wc}[2]{\begin{wconc}{#1}{}\setlength{\parindent}{1cm}#2\end{wconc}}
\newcommand{\thmcon}[1]{\begin{Theoremcon}{#1}\end{Theoremcon}}
\newcommand{\ex}[2]{\begin{Example}{#1}{}#2\end{Example}}
\newcommand{\dfn}[2]{\begin{Definition}[colbacktitle=red!75!black]{#1}{}#2\end{Definition}}
\newcommand{\dfnc}[2]{\begin{definition}[colbacktitle=red!75!black]{#1}{}#2\end{definition}}
\newcommand{\qs}[2]{\begin{question}{#1}{}#2\end{question}}
\newcommand{\pf}[2]{\begin{myproof}[#1]#2\end{myproof}}
\newcommand{\nt}[1]{\begin{note}#1\end{note}}

\newcommand*\circled[1]{\tikz[baseline=(char.base)]{
		\node[shape=circle,draw,inner sep=1pt] (char) {#1};}}
\newcommand\getcurrentref[1]{%
	\ifnumequal{\value{#1}}{0}
	{??}
	{\the\value{#1}}%
}
\newcommand{\getCurrentSectionNumber}{\getcurrentref{section}}
\newenvironment{myproof}[1][\proofname]{%
	\proof[\bfseries #1: ]%
}{\endproof}

\newcommand{\mclm}[2]{\begin{myclaim}[#1]#2\end{myclaim}}
\newenvironment{myclaim}[1][\claimname]{\proof[\bfseries #1: ]}{}

\newcounter{mylabelcounter}

\makeatletter
\newcommand{\setword}[2]{%
	\phantomsection
	#1\def\@currentlabel{\unexpanded{#1}}\label{#2}%
}
\makeatother




\tikzset{
	symbol/.style={
			draw=none,
			every to/.append style={
					edge node={node [sloped, allow upside down, auto=false]{$#1$}}}
		}
}


% deliminators
\DeclarePairedDelimiter{\abs}{\lvert}{\rvert}
\DeclarePairedDelimiter{\norm}{\lVert}{\rVert}

\DeclarePairedDelimiter{\ceil}{\lceil}{\rceil}
\DeclarePairedDelimiter{\floor}{\lfloor}{\rfloor}
\DeclarePairedDelimiter{\round}{\lfloor}{\rceil}

\newsavebox\diffdbox
\newcommand{\slantedromand}{{\mathpalette\makesl{d}}}
\newcommand{\makesl}[2]{%
\begingroup
\sbox{\diffdbox}{$\mathsurround=0pt#1\mathrm{#2}$}%
\pdfsave
\pdfsetmatrix{1 0 0.2 1}%
\rlap{\usebox{\diffdbox}}%
\pdfrestore
\hskip\wd\diffdbox
\endgroup
}
\newcommand{\dd}[1][]{\ensuremath{\mathop{}\!\ifstrempty{#1}{%
\slantedromand\@ifnextchar^{\hspace{0.2ex}}{\hspace{0.1ex}}}%
{\slantedromand\hspace{0.2ex}^{#1}}}}
\ProvideDocumentCommand\dv{o m g}{%
  \ensuremath{%
    \IfValueTF{#3}{%
      \IfNoValueTF{#1}{%
        \frac{\dd #2}{\dd #3}%
      }{%
        \frac{\dd^{#1} #2}{\dd #3^{#1}}%
      }%
    }{%
      \IfNoValueTF{#1}{%
        \frac{\dd}{\dd #2}%
      }{%
        \frac{\dd^{#1}}{\dd #2^{#1}}%
      }%
    }%
  }%
}
\providecommand*{\pdv}[3][]{\frac{\partial^{#1}#2}{\partial#3^{#1}}}
%  - others
\DeclareMathOperator{\Lap}{\mathcal{L}}
\DeclareMathOperator{\Var}{Var} % varience
\DeclareMathOperator{\Cov}{Cov} % covarience
\DeclareMathOperator{\E}{E} % expected

% Since the amsthm package isn't loaded

% I prefer the slanted \leq
\let\oldleq\leq % save them in case they're every wanted
\let\oldgeq\geq
\renewcommand{\leq}{\leqslant}
\renewcommand{\geq}{\geqslant}

% % redefine matrix env to allow for alignment, use r as default
% \renewcommand*\env@matrix[1][r]{\hskip -\arraycolsep
%     \let\@ifnextchar\new@ifnextchar
%     \array{*\c@MaxMatrixCols #1}}


%\usepackage{framed}
%\usepackage{titletoc}
%\usepackage{etoolbox}
%\usepackage{lmodern}


%\patchcmd{\tableofcontents}{\contentsname}{\sffamily\contentsname}{}{}

%\renewenvironment{leftbar}
%{\def\FrameCommand{\hspace{6em}%
%		{\color{myyellow}\vrule width 2pt depth 6pt}\hspace{1em}}%
%	\MakeFramed{\parshape 1 0cm \dimexpr\textwidth-6em\relax\FrameRestore}\vskip2pt%
%}
%{\endMakeFramed}

%\titlecontents{chapter}
%[0em]{\vspace*{2\baselineskip}}
%{\parbox{4.5em}{%
%		\hfill\Huge\sffamily\bfseries\color{myred}\thecontentspage}%
%	\vspace*{-2.3\baselineskip}\leftbar\textsc{\small\chaptername~\thecontentslabel}\\\sffamily}
%{}{\endleftbar}
%\titlecontents{section}
%[8.4em]
%{\sffamily\contentslabel{3em}}{}{}
%{\hspace{0.5em}\nobreak\itshape\color{myred}\contentspage}
%\titlecontents{subsection}
%[8.4em]
%{\sffamily\contentslabel{3em}}{}{}  
%{\hspace{0.5em}\nobreak\itshape\color{myred}\contentspage}



%%%%%%%%%%%%%%%%%%%%%%%%%%%%%%%%%%%%%%%%%%%
% TABLE OF CONTENTS
%%%%%%%%%%%%%%%%%%%%%%%%%%%%%%%%%%%%%%%%%%%

\usepackage{tikz}
\definecolor{doc}{RGB}{0,60,110}
\usepackage{titletoc}
\contentsmargin{0cm}
\titlecontents{chapter}[3.7pc]
{\addvspace{30pt}%
	\begin{tikzpicture}[remember picture, overlay]%
		\draw[fill=doc!60,draw=doc!60] (-7,-.1) rectangle (-0.9,.5);%
		\pgftext[left,x=-3.5cm,y=0.2cm]{\color{white}\Large\sc\bfseries Chapter\ \thecontentslabel};%
	\end{tikzpicture}\color{doc!60}\large\sc\bfseries}%
{}
{}
{\;\titlerule\;\large\sc\bfseries Page \thecontentspage
	\begin{tikzpicture}[remember picture, overlay]
		\draw[fill=doc!60,draw=doc!60] (2pt,0) rectangle (4,0.1pt);
	\end{tikzpicture}}%
\titlecontents{section}[3.7pc]
{\addvspace{2pt}}
{\contentslabel[\thecontentslabel]{2pc}}
{}
{\hfill\small \thecontentspage}
[]
\titlecontents*{subsection}[3.7pc]
{\addvspace{-1pt}\small}
{}
{}
{\ --- \small\thecontentspage}
[ \textbullet\ ][]

\makeatletter
\renewcommand{\tableofcontents}{%
	\chapter*{%
	  \vspace*{-20\p@}%
	  \begin{tikzpicture}[remember picture, overlay]%
		  \pgftext[right,x=15cm,y=0.2cm]{\color{doc!60}\Huge\sc\bfseries \contentsname};%
		  \draw[fill=doc!60,draw=doc!60] (13,-.75) rectangle (20,1);%
		  \clip (13,-.75) rectangle (20,1);
		  \pgftext[right,x=15cm,y=0.2cm]{\color{white}\Huge\sc\bfseries \contentsname};%
	  \end{tikzpicture}}%
	\@starttoc{toc}}
\makeatother


%From M275 "Topology" at SJSU
\newcommand{\id}{\mathrm{id}}
\newcommand{\taking}[1]{\xrightarrow{#1}}
\newcommand{\inv}{^{-1}}

%From M170 "Introduction to Graph Theory" at SJSU
\DeclareMathOperator{\diam}{diam}
\DeclareMathOperator{\ord}{ord}
\newcommand{\defeq}{\overset{\mathrm{def}}{=}}

%From the USAMO .tex files
\newcommand{\ts}{\textsuperscript}
\newcommand{\dg}{^\circ}
\newcommand{\ii}{\item}

% % From Math 55 and Math 145 at Harvard
% \newenvironment{subproof}[1][Proof]{%
% \begin{proof}[#1] \renewcommand{\qedsymbol}{$\blacksquare$}}%
% {\end{proof}}

\newcommand{\liff}{\leftrightarrow}
\newcommand{\lthen}{\rightarrow}
\newcommand{\opname}{\operatorname}
\newcommand{\surjto}{\twoheadrightarrow}
\newcommand{\injto}{\hookrightarrow}
\newcommand{\On}{\mathrm{On}} % ordinals
\DeclareMathOperator{\img}{im} % Image
\DeclareMathOperator{\Img}{Im} % Image
\DeclareMathOperator{\coker}{coker} % Cokernel
\DeclareMathOperator{\Coker}{Coker} % Cokernel
\DeclareMathOperator{\Ker}{Ker} % Kernel
\DeclareMathOperator{\rank}{rank}
\DeclareMathOperator{\Spec}{Spec} % spectrum
\DeclareMathOperator{\Tr}{Tr} % trace
\DeclareMathOperator{\pr}{pr} % projection
\DeclareMathOperator{\ext}{ext} % extension
\DeclareMathOperator{\pred}{pred} % predecessor
\DeclareMathOperator{\dom}{dom} % domain
\DeclareMathOperator{\ran}{ran} % range
\DeclareMathOperator{\Hom}{Hom} % homomorphism
\DeclareMathOperator{\Mor}{Mor} % morphisms
\DeclareMathOperator{\End}{End} % endomorphism

\newcommand{\eps}{\epsilon}
\newcommand{\veps}{\varepsilon}
\newcommand{\ol}{\overline}
\newcommand{\ul}{\underline}
\newcommand{\wt}{\widetilde}
\newcommand{\wh}{\widehat}
\newcommand{\vocab}[1]{\textbf{\color{blue} #1}}
\providecommand{\half}{\frac{1}{2}}
\newcommand{\dang}{\measuredangle} %% Directed angle
\newcommand{\ray}[1]{\overrightarrow{#1}}
\newcommand{\seg}[1]{\overline{#1}}
\newcommand{\arc}[1]{\wideparen{#1}}
\DeclareMathOperator{\cis}{cis}
\DeclareMathOperator*{\lcm}{lcm}
\DeclareMathOperator*{\argmin}{arg min}
\DeclareMathOperator*{\argmax}{arg max}
\newcommand{\cycsum}{\sum_{\mathrm{cyc}}}
\newcommand{\symsum}{\sum_{\mathrm{sym}}}
\newcommand{\cycprod}{\prod_{\mathrm{cyc}}}
\newcommand{\symprod}{\prod_{\mathrm{sym}}}
\newcommand{\Qed}{\begin{flushright}\qed\end{flushright}}
\newcommand{\parinn}{\setlength{\parindent}{1cm}}
\newcommand{\parinf}{\setlength{\parindent}{0cm}}
% \newcommand{\norm}{\|\cdot\|}
\newcommand{\inorm}{\norm_{\infty}}
\newcommand{\opensets}{\{V_{\alpha}\}_{\alpha\in I}}
\newcommand{\oset}{V_{\alpha}}
\newcommand{\opset}[1]{V_{\alpha_{#1}}}
\newcommand{\lub}{\text{lub}}
\newcommand{\del}[2]{\frac{\partial #1}{\partial #2}}
\newcommand{\Del}[3]{\frac{\partial^{#1} #2}{\partial^{#1} #3}}
\newcommand{\deld}[2]{\dfrac{\partial #1}{\partial #2}}
\newcommand{\Deld}[3]{\dfrac{\partial^{#1} #2}{\partial^{#1} #3}}
\newcommand{\lm}{\lambda}
\newcommand{\uin}{\mathbin{\rotatebox[origin=c]{90}{$\in$}}}
\newcommand{\usubset}{\mathbin{\rotatebox[origin=c]{90}{$\subset$}}}
\newcommand{\lt}{\left}
\newcommand{\rt}{\right}
\newcommand{\bs}[1]{\boldsymbol{#1}}
\newcommand{\exs}{\exists}
\newcommand{\st}{\strut}
\newcommand{\dps}[1]{\displaystyle{#1}}

\newcommand{\sol}{\setlength{\parindent}{0cm}\textbf{\textit{Solution:}}\setlength{\parindent}{1cm} }
\newcommand{\solve}[1]{\setlength{\parindent}{0cm}\textbf{\textit{Solution: }}\setlength{\parindent}{1cm}#1 \Qed}

% Things Lie
\newcommand{\kb}{\mathfrak b}
\newcommand{\kg}{\mathfrak g}
\newcommand{\kh}{\mathfrak h}
\newcommand{\kn}{\mathfrak n}
\newcommand{\ku}{\mathfrak u}
\newcommand{\kz}{\mathfrak z}
\DeclareMathOperator{\Ext}{Ext} % Ext functor
\DeclareMathOperator{\Tor}{Tor} % Tor functor
\newcommand{\gl}{\opname{\mathfrak{gl}}} % frak gl group
\renewcommand{\sl}{\opname{\mathfrak{sl}}} % frak sl group chktex 6

% More script letters etc.
\newcommand{\SA}{\mathcal A}
\newcommand{\SB}{\mathcal B}
\newcommand{\SC}{\mathcal C}
\newcommand{\SF}{\mathcal F}
\newcommand{\SG}{\mathcal G}
\newcommand{\SH}{\mathcal H}
\newcommand{\OO}{\mathcal O}

\newcommand{\SCA}{\mathscr A}
\newcommand{\SCB}{\mathscr B}
\newcommand{\SCC}{\mathscr C}
\newcommand{\SCD}{\mathscr D}
\newcommand{\SCE}{\mathscr E}
\newcommand{\SCF}{\mathscr F}
\newcommand{\SCG}{\mathscr G}
\newcommand{\SCH}{\mathscr H}

% Mathfrak primes
\newcommand{\km}{\mathfrak m}
\newcommand{\kp}{\mathfrak p}
\newcommand{\kq}{\mathfrak q}

% number sets
\newcommand{\RR}[1][]{\ensuremath{\ifstrempty{#1}{\mathbb{R}}{\mathbb{R}^{#1}}}}
\newcommand{\NN}[1][]{\ensuremath{\ifstrempty{#1}{\mathbb{N}}{\mathbb{N}^{#1}}}}
\newcommand{\ZZ}[1][]{\ensuremath{\ifstrempty{#1}{\mathbb{Z}}{\mathbb{Z}^{#1}}}}
\newcommand{\QQ}[1][]{\ensuremath{\ifstrempty{#1}{\mathbb{Q}}{\mathbb{Q}^{#1}}}}
\newcommand{\CC}[1][]{\ensuremath{\ifstrempty{#1}{\mathbb{C}}{\mathbb{C}^{#1}}}}
\newcommand{\PP}[1][]{\ensuremath{\ifstrempty{#1}{\mathbb{P}}{\mathbb{P}^{#1}}}}
\newcommand{\HH}[1][]{\ensuremath{\ifstrempty{#1}{\mathbb{H}}{\mathbb{H}^{#1}}}}
\newcommand{\FF}[1][]{\ensuremath{\ifstrempty{#1}{\mathbb{F}}{\mathbb{F}^{#1}}}}
% expected value
\newcommand{\EE}{\ensuremath{\mathbb{E}}}
\newcommand{\charin}{\text{ char }}
\DeclareMathOperator{\sign}{sign}
\DeclareMathOperator{\Aut}{Aut}
\DeclareMathOperator{\Inn}{Inn}
\DeclareMathOperator{\Syl}{Syl}
\DeclareMathOperator{\Gal}{Gal}
\DeclareMathOperator{\GL}{GL} % General linear group
\DeclareMathOperator{\SL}{SL} % Special linear group

%---------------------------------------
% BlackBoard Math Fonts :-
%---------------------------------------

%Captital Letters
\newcommand{\bbA}{\mathbb{A}}	\newcommand{\bbB}{\mathbb{B}}
\newcommand{\bbC}{\mathbb{C}}	\newcommand{\bbD}{\mathbb{D}}
\newcommand{\bbE}{\mathbb{E}}	\newcommand{\bbF}{\mathbb{F}}
\newcommand{\bbG}{\mathbb{G}}	\newcommand{\bbH}{\mathbb{H}}
\newcommand{\bbI}{\mathbb{I}}	\newcommand{\bbJ}{\mathbb{J}}
\newcommand{\bbK}{\mathbb{K}}	\newcommand{\bbL}{\mathbb{L}}
\newcommand{\bbM}{\mathbb{M}}	\newcommand{\bbN}{\mathbb{N}}
\newcommand{\bbO}{\mathbb{O}}	\newcommand{\bbP}{\mathbb{P}}
\newcommand{\bbQ}{\mathbb{Q}}	\newcommand{\bbR}{\mathbb{R}}
\newcommand{\bbS}{\mathbb{S}}	\newcommand{\bbT}{\mathbb{T}}
\newcommand{\bbU}{\mathbb{U}}	\newcommand{\bbV}{\mathbb{V}}
\newcommand{\bbW}{\mathbb{W}}	\newcommand{\bbX}{\mathbb{X}}
\newcommand{\bbY}{\mathbb{Y}}	\newcommand{\bbZ}{\mathbb{Z}}

%---------------------------------------
% MathCal Fonts :-
%---------------------------------------

%Captital Letters
\newcommand{\mcA}{\mathcal{A}}	\newcommand{\mcB}{\mathcal{B}}
\newcommand{\mcC}{\mathcal{C}}	\newcommand{\mcD}{\mathcal{D}}
\newcommand{\mcE}{\mathcal{E}}	\newcommand{\mcF}{\mathcal{F}}
\newcommand{\mcG}{\mathcal{G}}	\newcommand{\mcH}{\mathcal{H}}
\newcommand{\mcI}{\mathcal{I}}	\newcommand{\mcJ}{\mathcal{J}}
\newcommand{\mcK}{\mathcal{K}}	\newcommand{\mcL}{\mathcal{L}}
\newcommand{\mcM}{\mathcal{M}}	\newcommand{\mcN}{\mathcal{N}}
\newcommand{\mcO}{\mathcal{O}}	\newcommand{\mcP}{\mathcal{P}}
\newcommand{\mcQ}{\mathcal{Q}}	\newcommand{\mcR}{\mathcal{R}}
\newcommand{\mcS}{\mathcal{S}}	\newcommand{\mcT}{\mathcal{T}}
\newcommand{\mcU}{\mathcal{U}}	\newcommand{\mcV}{\mathcal{V}}
\newcommand{\mcW}{\mathcal{W}}	\newcommand{\mcX}{\mathcal{X}}
\newcommand{\mcY}{\mathcal{Y}}	\newcommand{\mcZ}{\mathcal{Z}}


%---------------------------------------
% Bold Math Fonts :-
%---------------------------------------

%Captital Letters
\newcommand{\bmA}{\boldsymbol{A}}	\newcommand{\bmB}{\boldsymbol{B}}
\newcommand{\bmC}{\boldsymbol{C}}	\newcommand{\bmD}{\boldsymbol{D}}
\newcommand{\bmE}{\boldsymbol{E}}	\newcommand{\bmF}{\boldsymbol{F}}
\newcommand{\bmG}{\boldsymbol{G}}	\newcommand{\bmH}{\boldsymbol{H}}
\newcommand{\bmI}{\boldsymbol{I}}	\newcommand{\bmJ}{\boldsymbol{J}}
\newcommand{\bmK}{\boldsymbol{K}}	\newcommand{\bmL}{\boldsymbol{L}}
\newcommand{\bmM}{\boldsymbol{M}}	\newcommand{\bmN}{\boldsymbol{N}}
\newcommand{\bmO}{\boldsymbol{O}}	\newcommand{\bmP}{\boldsymbol{P}}
\newcommand{\bmQ}{\boldsymbol{Q}}	\newcommand{\bmR}{\boldsymbol{R}}
\newcommand{\bmS}{\boldsymbol{S}}	\newcommand{\bmT}{\boldsymbol{T}}
\newcommand{\bmU}{\boldsymbol{U}}	\newcommand{\bmV}{\boldsymbol{V}}
\newcommand{\bmW}{\boldsymbol{W}}	\newcommand{\bmX}{\boldsymbol{X}}
\newcommand{\bmY}{\boldsymbol{Y}}	\newcommand{\bmZ}{\boldsymbol{Z}}
%Small Letters
\newcommand{\bma}{\boldsymbol{a}}	\newcommand{\bmb}{\boldsymbol{b}}
\newcommand{\bmc}{\boldsymbol{c}}	\newcommand{\bmd}{\boldsymbol{d}}
\newcommand{\bme}{\boldsymbol{e}}	\newcommand{\bmf}{\boldsymbol{f}}
\newcommand{\bmg}{\boldsymbol{g}}	\newcommand{\bmh}{\boldsymbol{h}}
\newcommand{\bmi}{\boldsymbol{i}}	\newcommand{\bmj}{\boldsymbol{j}}
\newcommand{\bmk}{\boldsymbol{k}}	\newcommand{\bml}{\boldsymbol{l}}
\newcommand{\bmm}{\boldsymbol{m}}	\newcommand{\bmn}{\boldsymbol{n}}
\newcommand{\bmo}{\boldsymbol{o}}	\newcommand{\bmp}{\boldsymbol{p}}
\newcommand{\bmq}{\boldsymbol{q}}	\newcommand{\bmr}{\boldsymbol{r}}
\newcommand{\bms}{\boldsymbol{s}}	\newcommand{\bmt}{\boldsymbol{t}}
\newcommand{\bmu}{\boldsymbol{u}}	\newcommand{\bmv}{\boldsymbol{v}}
\newcommand{\bmw}{\boldsymbol{w}}	\newcommand{\bmx}{\boldsymbol{x}}
\newcommand{\bmy}{\boldsymbol{y}}	\newcommand{\bmz}{\boldsymbol{z}}

%---------------------------------------
% Scr Math Fonts :-
%---------------------------------------

\newcommand{\sA}{{\mathscr{A}}}   \newcommand{\sB}{{\mathscr{B}}}
\newcommand{\sC}{{\mathscr{C}}}   \newcommand{\sD}{{\mathscr{D}}}
\newcommand{\sE}{{\mathscr{E}}}   \newcommand{\sF}{{\mathscr{F}}}
\newcommand{\sG}{{\mathscr{G}}}   \newcommand{\sH}{{\mathscr{H}}}
\newcommand{\sI}{{\mathscr{I}}}   \newcommand{\sJ}{{\mathscr{J}}}
\newcommand{\sK}{{\mathscr{K}}}   \newcommand{\sL}{{\mathscr{L}}}
\newcommand{\sM}{{\mathscr{M}}}   \newcommand{\sN}{{\mathscr{N}}}
\newcommand{\sO}{{\mathscr{O}}}   \newcommand{\sP}{{\mathscr{P}}}
\newcommand{\sQ}{{\mathscr{Q}}}   \newcommand{\sR}{{\mathscr{R}}}
\newcommand{\sS}{{\mathscr{S}}}   \newcommand{\sT}{{\mathscr{T}}}
\newcommand{\sU}{{\mathscr{U}}}   \newcommand{\sV}{{\mathscr{V}}}
\newcommand{\sW}{{\mathscr{W}}}   \newcommand{\sX}{{\mathscr{X}}}
\newcommand{\sY}{{\mathscr{Y}}}   \newcommand{\sZ}{{\mathscr{Z}}}


%---------------------------------------
% Math Fraktur Font
%---------------------------------------

%Captital Letters
\newcommand{\mfA}{\mathfrak{A}}	\newcommand{\mfB}{\mathfrak{B}}
\newcommand{\mfC}{\mathfrak{C}}	\newcommand{\mfD}{\mathfrak{D}}
\newcommand{\mfE}{\mathfrak{E}}	\newcommand{\mfF}{\mathfrak{F}}
\newcommand{\mfG}{\mathfrak{G}}	\newcommand{\mfH}{\mathfrak{H}}
\newcommand{\mfI}{\mathfrak{I}}	\newcommand{\mfJ}{\mathfrak{J}}
\newcommand{\mfK}{\mathfrak{K}}	\newcommand{\mfL}{\mathfrak{L}}
\newcommand{\mfM}{\mathfrak{M}}	\newcommand{\mfN}{\mathfrak{N}}
\newcommand{\mfO}{\mathfrak{O}}	\newcommand{\mfP}{\mathfrak{P}}
\newcommand{\mfQ}{\mathfrak{Q}}	\newcommand{\mfR}{\mathfrak{R}}
\newcommand{\mfS}{\mathfrak{S}}	\newcommand{\mfT}{\mathfrak{T}}
\newcommand{\mfU}{\mathfrak{U}}	\newcommand{\mfV}{\mathfrak{V}}
\newcommand{\mfW}{\mathfrak{W}}	\newcommand{\mfX}{\mathfrak{X}}
\newcommand{\mfY}{\mathfrak{Y}}	\newcommand{\mfZ}{\mathfrak{Z}}
%Small Letters
\newcommand{\mfa}{\mathfrak{a}}	\newcommand{\mfb}{\mathfrak{b}}
\newcommand{\mfc}{\mathfrak{c}}	\newcommand{\mfd}{\mathfrak{d}}
\newcommand{\mfe}{\mathfrak{e}}	\newcommand{\mff}{\mathfrak{f}}
\newcommand{\mfg}{\mathfrak{g}}	\newcommand{\mfh}{\mathfrak{h}}
\newcommand{\mfi}{\mathfrak{i}}	\newcommand{\mfj}{\mathfrak{j}}
\newcommand{\mfk}{\mathfrak{k}}	\newcommand{\mfl}{\mathfrak{l}}
\newcommand{\mfm}{\mathfrak{m}}	\newcommand{\mfn}{\mathfrak{n}}
\newcommand{\mfo}{\mathfrak{o}}	\newcommand{\mfp}{\mathfrak{p}}
\newcommand{\mfq}{\mathfrak{q}}	\newcommand{\mfr}{\mathfrak{r}}
\newcommand{\mfs}{\mathfrak{s}}	\newcommand{\mft}{\mathfrak{t}}
\newcommand{\mfu}{\mathfrak{u}}	\newcommand{\mfv}{\mathfrak{v}}
\newcommand{\mfw}{\mathfrak{w}}	\newcommand{\mfx}{\mathfrak{x}}
\newcommand{\mfy}{\mathfrak{y}}	\newcommand{\mfz}{\mathfrak{z}}


\title{MAS 5145 Lecture Notes - Fall 2024}
\author{Nickolas Arustamyan}
\date{\today}

\begin{document}
\maketitle
\tableofcontents
\chapter{Review}
\section{08/20/2024}
\dfn{Vector Spaces}{A Vector Space is a nonempty set V with two operations, vector addition and scaler multiplication. These operations must satisfy a bunch of axioms, most important of which is $u,v \in V \implies u+v \in V$ and for $\alpha \in \mathbb{F}, \alpha v \in V$.}
\dfn{Subspace}{A Subspace W of a vector space V is a nonempty subset of V with the same operations as V. }

\mprop{}{The intersection of any collection of subspaces $W_j$ of V is itself a subspace of V}
\begin{myproof}
In order to be a subspace, we must prove that the intersection is nonempty and that it is closed under the operations of V. Clearly, since each $W_j$ is a subspace, they must each contain the zero element. Hence, the intersection must as well and hence, the intersection is nonempty. For any elements $u, v \in \bigcap W_j = W$, we know that a linear combination $\alpha u + \beta v \in W_j$ for each $W_j$ since they are each subspaces and hence closed under the vector operations. This means that $\alpha u + \beta v \in W$ and hence $W$ is a subspace.
\end{myproof}

\dfn{Direct Sum}{Given two subspaces $W_1$ and $W_2$ of $V$, if $W_1\cap W_2 = \{0\}$, then $W_1+W_2$ is a direct sum of $W_1$ and $W_2$. For a collection of subspaces, we have a direct sum of $W_i \bigcap_{j \neq i} W_j = \{0\}$ }

\section{08/22/2024}
\dfn{Linear Combination}{Let $V$ be a vector space and $B = \{v_1, ..., v_k\} \subset V$. A linear combination of $B$ is a vector of the form $v = \sum c_iv_i (c_i \in \mathbb{F}) $. }

\dfn{Spanning Set}{$S \subseteq V$ is called a spanning set if $span(S) = V$. }

\dfn{Linear Independence and Dependence}{$B\subset V$ is called linearly dependent if there exists $c_1, ..., c_n$ not all $0$ such that $\sum b_i c_i = 0$. Otherwise we say that $B$ is linearly independent.}

\dfn{Basis}{$S \subseteq V$ is called a basis of $V$ if it is linearly independent and spanning.}

\thm{}{Every Vector Space has a basis}
\begin{myproof}
    In the finite dimensional case, we know that $V = span(\{v_1, ..., v_n\})$ for some set $v_1,..., v_n$. If the spanning set is linearly independent, then we have a basis. Otherwise, remove linearly dependent vectors and recheck until we have a linearly independent set, which is thus a basis. In the infinite dimensional case, we must use Zorns Lemma but it is true that we can find the basis.
\end{myproof}

\qs{}{Every spanning set contains a basis}
\begin{myproof}
    Let $B = \{v_1, ..., v_n\}$ such that $B$ is a spanning set. If $B$ is linearly independent, the basis is itself. Otherwise, there must be some vector $v_k$ that can be written as a linear combination of the other vectors. We can remove $v_k$ and recheck the new $B$ to see if it is linearly independent. This process must terminate and when it does, the final set will be linearly independent by construction. Hence, that final set will be a basis.
\end{myproof}

\qs{}{Every linearly independent set can be extended to a basis}
\begin{myproof}
    Let $B = \{v_1, ..., v_n\}$ such that $B$ is linearly independent. If $span(B) = V$, then we have a basis. Otherwise, there must be some vector $v \in V$ such that $v \not\in span(B)$. Append $v$ to $B$ and recheck if it is a spanning set. If not, repeat the process until we have a spanning set. At that point we will have a basis.
\end{myproof}

\qs{}{Suppose $A = \{v_1, ... , v_k\}$ is linearly independent and $B = \{w_1, ... , w_m\}$ is a spanning set. Then $k \leq m$. }
\begin{myproof}
    We know that every linearly independent set can be extended to form a basis. This means that one can turn the LI set into one that also spans only by adding vectors to it. Similarly, every spanning set contains a basis implies that one can turn a spanning set into one that also is LI only by removing vectors from it. Together, these imply that the cardinality of any spanning set must be greater than or equal to that of any LI set. Hence, $k \leq m$.
\end{myproof}

\dfn{Dimension of a Vector Space}{Let $S$ be a basis for a vector space $V$. Then $dim(V)$ is the cardinality of $S$.}

\mlenma{}{Let $dim(V) = n < \infty$. Every $n$ LI vectors form a basis.}
\begin{myproof}
    Let $\{v_1, v_2, ..., v_n\}$ be a LI set. Then we can extend it to a basis $B = \{v_1, v_2, ..., v_n, u_1, ..., u_k\}$. But we know that the dimension of $V$ is $n$ and since $B$ is a basis of $V$, then $dim(V) = n+k$. Hence $n = n+k$ which implies $k = 0$ and the original set was a basis. 
\end{myproof}

\mlenma{}{Let $dim(V) = n < \infty$. Every $n$ spanning vectors form a basis}
\begin{myproof}
    Let $\{v_1, v_2, ..., v_n\}$ be a spanning set. Then we can select vectors from it to form a basis $B = \{v_{i_1}, v_{i_2}, ..., v_{i_k} \}$. But we know that the dimension of $V$ is $n$ and since $B$ is a basis of $V$, then $dim(V) = k$. Hence $n = k$ which implies the original set was a basis.
\end{myproof}
If $B$ is a basis of $V$, then every vector in $V$ has a unique representation as a linear combination of vectors in $B$. 
\section{08/27/2024}
Let $W_1, W_2$ be subspaces of $V$. Then the following are equivilant 
\begin{enumerate}
    \item $W_1 \oplus W_2$
    \item For all $v\in W_1 + W_2$ there exists unique $w_1 \in W_1, w_2 \in W_2$ such that $v = w_1 + w_2$
    \item $w_1 + w_2 = 0 \implies w_1 = w_2 = 0$
    \item The union of a basis for $W_1$ one for $W_2$ is basis of $W_1 + W_2$
\end{enumerate}

\begin{myproof}
    The first three all follow from definition of the direct sum. The fourth equivilance can be seen by letting $A = \{u_1, u_2, ..., u_k\}$ be a basis for $W_1$ and $B = \{v_1, v_2, ..., v_n\}$ be a basis for $W_2$. Clearly, $span(A\cup B) = W_1+W_2$. Linear indepence can be seen since the two sets themselves are basis and hence linearly independent. Thus, we have a basis of $W_1 + W_2$. 
\end{myproof}

\mlenma{}{If $W$ is a subspace of $V$, then there exists a subspace $U$ of $V$ such that $V = U \oplus W$. $U$ is called a complement of $V$. }
\begin{myproof}
    Let $\{w_1, w_2, ..., w_k\}$ be a basis of $W$. Extend it to be a basis for $V, \{w_1, ..., w_k, u_1, ..., u_n\}$. Set $U = span(\{u_1, ..., u_n\})$. Thus, $V = U \oplus W$.
\end{myproof}

\thm{Dimension Formula for Subspaces}{Let $W, U$ be finite dimensional subspaces of $V$. Then $dim(W+U) = dim(W) + dim(U) - dim(W \cap U)$.}
\begin{myproof}
    Let $B_1 = \{v_1, ..., v_n\}$ be a basis for $W\cap U$. Then we can extend $B_1$ to form a basis of $W$ and get $B_2 = \{v_1, ..., v_n, w_1, ..., w_k\}$. Similarly, we can extend $B_1$ to form a basis for $U$ and get $B_3 = \{v_1, ..., v_n, u_1, ..., u_l\}$. Thus $dim(W) + dim(U) + dim(W \cap U) = n + k + n + l - n = n+k+l$. Now, we claim that $B = \{v_1, ..., v_n, w_1, ..., w_k, u_1, ..., u_l\}$ is a basis for $W+U$. Clearly, $span(B) = W+U$. To verify linear independence, we first set $\sum a_i v_i + \sum b_j w_j + \sum c_m u_m = 0$. This means that $-\sum a_i v_i = \sum b_j w_j + \sum c_m u_m$. Since the left hand side is in $W$ and the right hand side is in $U$, then they must  be in the intersction. This means that there are some $d_i \in \mathbb{F}$ such that $\sum d_i v_i = -\sum c_m u_m$. Since $B_1$ is a linearly independent set, this implies that $c_m = 0$. From this, it follows $a_i, b_j = 0$ and hence the set $B$ is linearly independent and a basis. Thus, $dim(W+U) = n+k+l$. 
\end{myproof}

\dfn{Linear Transformation}{Let $V, W$ be two vector spaces over $\mathbb{F}$. A map $\alpha: V \rightarrow W$ is linear if $\alpha(au+bv) = a\alpha(u) + b\alpha(v)$}

\mprop{}{If $\alpha$ is linear, then:
\begin{itemize}
\item $ker(\alpha) = \{ v\in V : \alpha(v) = 0\}$ is a subspace of $V$.
\item $\alpha(0) = 0$
\item $im(\alpha) = \{\alpha(v) : v\in V  \}$ is a subspace of $W$.
\end{itemize}}
\begin{myproof}
    NEED TO DO
\end{myproof}

\dfn{Injective or One-to-One}{A linear transformation $\alpha$ is called injective or one to one if $\alpha(x) = \alpha(y) $ if and only if $x=y$}

\qs{}{A linear transformation $\alpha$ is injective if and only if $ker(\alpha) = \{0\}$. }


\begin{myproof}
    Assume that $\alpha$ is injective. Now, suppose that $x \in ker(\alpha)$. This means that $\alpha(x) = 0$. But we know that $\alpha(0) = 0$. Since $\alpha$ is injective and $\alpha(x) = \alpha(0) = 0$, then $x = 0$. Hence $ker(\alpha) = \{0\}$. This has proved one direction. For the other direction, assume that $ker(\alpha) = \{0\}$. Now assume that $\alpha(x) = \alpha(y)$. This means that $\alpha(x) - \alpha(y) = 0$ and hence $x- y \in ker(\alpha)$. But this implies that $x=y$. Hence, when $\alpha(x) = \alpha(y)$, $x=y$. Thus, $\alpha$ is injective. 
    \end{myproof}


\dfn{Surjective or onto}{A linear transformation $\alpha$ is called surjective or onto if $im(\alpha) = W$.}

\dfn{Bijective}{A linear transformation $\alpha$ is called bijective it is both injective and surjective.}

\mprop{}{Let $\alpha: V \rightarrow W$ be linear:
\begin{enumerate}
\item $\alpha$ is injective if and only if $ker(\alpha) = 0$
\item If $\alpha$ is bijective, $\alpha^{-1}$ is linear
\item If $\alpha$ is injective and $S$ is LI, then $\alpha(S)$ is LI
\item If $\alpha$ is surjective and $span(S) = V, span(\alpha(S)) = W$. 
\item If $\alpha$ is bijective, $\alpha$ maps a basis of $V$ to a basis of $W$. 
\end{enumerate}}


\begin{myproof}
    NEED TO DO
\end{myproof}

\mprop{}{If $\alpha: V \rightarrow W$ and $\beta: W \rightarrow U$ are both linear, then so is $\beta\alpha$. }

\begin{myproof}
    NEED TO DO
\end{myproof}

\dfn{Nullity}{We define $nullity(\alpha) = dim(ker(\alpha))$.
}

\dfn{Rank}{We define $rank(\alpha) = dim(im(\alpha))$.
}

\thm{Dimension Formula for Linear Transformations}{If $V$ and $W$ are finite dimensional vector spaces, then if $\alpha$ is linear, $dim(V) = rank(\alpha) + nullity(\alpha)$.}

\begin{myproof}
    Since $ker(\alpha)$ is a subspace, we know it must have a basis. Let $B = {v_1, v_2, ..., v_k}$ be a basis for $ker(\alpha)$. We know that the kernal is finite dimensional since it is a subspace of a finite dimensional space $V$. Now, extend $B$ to form a basis for $V$. Let $dim(V) = n$. Hence, we get $A = {v_1, ..., v_k, v_{k+1}, ..., v_n}$. Now, $im(\alpha) = span(\alpha(A)) = span(\alpha(v_1), ..., \alpha(v_k), \alpha(v_{k+1}), ..., \alpha(v_n)) = span(\alpha(v_{k+1}), ..., \alpha(v_n)) = span(A \setminus B)$. We now know that $A\setminus B = C$ is a spanning set of $im(\alpha)$. To see that $C$ is linearly independent, we set $\sum_{i=1}^{n-k} d_i\alpha(v_i) = 0$. Since $\alpha$ is linear, we can take out the transformation and get $\alpha(\sum_{i=1}^{n-k} d_iv_i) = 0$. But this means that $\sum_{i=1}^{n-k} d_iv_i \in ker(\alpha)$ which cannot be the case unless all $d_i = 0$. Hence, $C$ is linearly independent and thus a basis for $im(\alpha)$. Thus $rank(\alpha) + im(\alpha) = k + (n-k) = n = dim(V)$.
\end{myproof}

\section{08/29/2024}

\mlenma{}{Suppose $dim(V) = dim(W) = n < \infty$ and $T:V \rightarrow W$ is a linear map. Then $T$ is injective if and only if it is surjective.}

\begin{myproof}
    Assume that $T$ is injective. Then $ker(T) = \{0\}$ and hence $nullity(T) = 0$. Thus, by the Dimension Formula for Linear Transformations, we know that $rank(T) = dim(V) - nullity(T) = n-0=n$. Hence $rank(T) = n$ and thus, $T$ is surjective. Now, assume instead that $T$ is surjective. Similarly, we know that $nullity(T) = 0$ and thus, $T$ is injective. 
\end{myproof}

\thm{}{    Say $T:V \rightarrow W$ and $S:W \rightarrow Y$ with $T, S$ linear. Then 
\begin{enumerate}
    \item $nullity(ST) \leq nullity(T) + nullity(S)$
    \item $rank(T) + rank(S) - dim(W) \leq rank(ST) \leq  min(rank(S), rank(T))$
\end{enumerate}}

\begin{myproof}
    Since $ker(ST)$ is a subspace, we know it has a basis and hence, let $B = \{c_1, ..., c_g\}$ to be a basis for the kernal. This is the set of all vectors $c\in V$ such that $S(T(c)) = 0$. These specific $T(v)$ form a subset of $ker(S)$. Call that set $G$. Then $G \subseteq ker(S)$ and hence $ker(ST) \subseteq ker(S)$. This implies that $nullity(ST) \leq nullity(S) \leq nullity(S) + nullity(T)$.
    \newline
    To prove the second item, NEED TO DO
\end{myproof}

\dfn{Homomorphisms}{Let $V, W$ be two vector spaces. The set of all linear transformations from $V$ to $W$ is called $Hom(V,W)$ or $L(V, W)$. }
\mclm{}{$Hom(V, W)$ is a vector space and $dim(Hom(V, W)) = dim(V)\cdot dim(W)$.}

\chapter{Eigen Things}
\section{09/05/2024}
\dfn{Eigenvalue}{Let $\alpha \in End(V)$. Then $\lambda$ is an eigenvalue if there exists $v\neq 0$ such that $\alpha(v) = \lambda v$. We call $v$ an eigenvector of $\alpha$. }

\dfn{Spectrum}{We call the set of all eigenvalues of $\alpha$ the spectrum of $\alpha$ and denote it $spec(\alpha)$. }

\dfn{Eigenspace}{We call $E_\lambda(\alpha) = \{v : \alpha(v) = \lambda v\}$ the eignespace of $\alpha$ with respsect to $\lambda$. It is a subspace of $V$. }
\dfn{Spectral Radius}{The spectral radius of $\alpha$ is $\rho(\alpha) = sup\{|\lambda|: \lambda \in spec(\alpha)\}$.}

\dfn{Charectoristic Polynomial}{Let $\alpha \in End(V)$ and $dim(V) = n$ be finite. Then the charectoristic polynomial of $\alpha$ is defined by $p(t) = det(tI-A)$ where $A$ is a matrix representationof $\alpha$ with respect to a basis $D$. $p(t) = t^n+a_{n-1}t^{n-1} + ... + a_1t+a_0$ and is independent of the choice of basis.}
\qs{}{Prove that $p(t)$ is independent of the choice of basis}
\pf{}{If $B$ and $C$ are two representations of $\alpha$ with respect to basis $D, E$ respectivly, then $B = S^{-1}CS$ for some invertible matrix $S$ (the change of basis matrix). Then $det(tI-B) = det(tS^{-1}S-S^{-1}CS) = det(S^{-1}(tI-C)S) = det(tI-C)$}

\qs{}{Prove that if $\lambda \in spec(\alpha)$, then $p_\alpha(\lambda) = 0$. }
\pf{}{If $\lambda \in spec(\alpha)$, then $(\lambda I - A)\hat{x} = 0$ has a nontrivial solution. Thus $(\lambda I - A)$ is non invertible and hence $det(\lambda I - A) = p_\alpha(\lambda) = 0$. }
\thm{Spectral Mapping Theorem for Polynomials}{Let $\alpha \in End(V)$ and $p(t)$ be a polynomial. Then $spec(p(\alpha)) = \{p(\lambda) : \lambda \in spec(\alpha)\}$ where $p(\alpha) = \sum a_i\alpha^i = a_0I+a_1\alpha + ... + a_n\alpha^n \in End(V)$. }
\pf{}{Let $\lambda \in spec(\alpha)$. Then there exists $v \neq 0$ such that $\alpha(v) = \lambda v$. This means that for all $k \in \bbN, \alpha^k(v) = \lambda^kv$. Thus $p(\alpha)(v) = \sum a_i \alpha^i(v) =\sum a_i \lambda^iv  = v\sum a_i \lambda^i = p(\lambda)v$. Since $v\neq 0$, $p(\lambda) \in spec(p(\alpha))$. This proves one direction. To show the other, let $\mu \in spec(p(\alpha))$. Consider $q(t) = p(t)-\mu = a_n(t-\lambda_1)(t-\lambda_2)...(t-\lambda_n)$. This means that $q(\alpha) = p(\alpha) - \mu I$. Since $\mu \in spec(p(\alpha))$, we know that $p(\alpha) -\mu I$ is not invertible. But this means for some $i_0, \alpha-\lambda_{i_0} I$ is not invertible. Thus, $\lambda_{i_0}\in spec(\alpha)$. Since $q(\lambda_{i_0})=0, \mu = p(\lambda_{i_0})$. }

\section{09/10/2024}
\thm{}{Let $\alpha \in End(V)$ and $p(t) \in P(\mathbb{F})$. Then $p(spec(\alpha)) = \{p(\lambda): \lambda \in spec(\alpha)\} \subseteq spec(p(\alpha))$}
\qs{}{Let $\alpha_1, ..., \alpha_k \in End(V)$ such that $\alpha_i\cdot \alpha_j = \alpha_j\cdot \alpha_i$ for all $i,j$. Then $\alpha = \prod_j \alpha_j$ is invertible if and only if every $\alpha_i$ is invertible. }
\mprop{}{Suppose $dim(V) < \infty$ and $\alpha, \beta \in End(V)$. Then $spec(\alpha\beta) = spec(\beta\alpha)$. If $dim(V) = \infty$ then $spec(\alpha\beta) \cup \{0\} = spec(\beta\alpha) \cup \{0\}$}

\thm{Cayley Hamilton Theorem}{Suppose $dim(V) = n < \infty$ and $\alpha \in End(V)$. Then $p(\alpha) = 0$ where $p(t)$ is the charectoristic polynomial of $\alpha$, $p(t) = det(tI - A)$.}

\dfn{Minimal Polynomial}{Let $\alpha \in End(V)$ and $dim(V) = n < \infty$. Then the minimal polynomial, $m(t)$ of $\alpha$ is the monic polynomial of smallest positive degree such that $m(\alpha) = 0$.  }

\mprop{Minimal Polynomial divides the charectoristic polynomial}{Let $m(t)$ be the minimal polynomial of $\alpha$ and $p(t)$ such that $p(\alpha) = 0$. Then $m(t)|p(t)$}
\pf{}{Clearly, $deg(m) \leq deg(p)$. Doing long division, we see that $p(t) = m(t)g(t)+h(t)$. So this means that $p(\alpha) = h(\alpha)$. But this means that $deg(h) > deg(m)$, which is a contradiction. Thus $h(t) = 0$ and hence, $m | p$. }
\mprop{}{Let $m(t)$ be the minimal polynomial of $\alpha \in End(V)$. Then $spec(\alpha) = \{\lambda \in \mathbb{F} : m(\lambda) = 0\}$. }
\pf{}{Recall that $spec(\alpha) = \{\lambda \in \mathbb{F} : p(\lambda) = 0\}$ where $p$ is the charectoristic polynomial of $\alpha$. Since $m|p$, if $m(\lambda) = 0$, then $p(\lambda) = 0$. So this means that $\{\lambda \in \mathbb{F} : m(\lambda) = 0\} \subseteq spec(\alpha)$. Conversly, if $\lambda \in spec(\alpha)$, then by the spectral mapping theorem, $m(\lambda) \in spec(m(\alpha)) = spec(0) = \{0\}$. This means that $m(\lambda) = 0$. Hence $ spec(\alpha) \subseteq  \{\lambda \in \mathbb{F} : m(\lambda) = 0\}$ and so they are equal. }

\cor{}{If $\lambda_1, ..., \lambda_k$ are all the distinct eigenvalues of $\alpha$, then $m(t) = (t-\lambda_1)\cdot (t-\lambda_2) \cdot ... \cdot (t-\lambda_k)h(t)$ for some polynomial $h(t)$. If $dim(V) = k$, then $m(t) = (t-\lambda_1)\cdot (t-\lambda_2) \cdot ... \cdot (t-\lambda_k)$.}

\dfn{}{Let $\alpha \in End(V)$ and $dim(V) = n <\infty$. Then $\alpha$ is diagonalizable if there exists an ordered basis $B = \{v_1, ..., v_k\}$ of $V$ such that $[\alpha]_B$ is diagonal.}

\section{09/12/2024}
Let $\alpha \in End(V)$
\begin{enumerate}
    \item $\alpha$ is diagonalizable if there exists a basis $B$ for $V$ such that $[\alpha]_B$ is diagonal
    \item $\alpha$ is diagonalizable if and only if $B$ is only eigenvectors
    \item $\alpha$ is diagnalizable implies that $p(t)$ splits
    \item $A \in M_{n\times n}(\bbF)$ is diagnalizable if and only if there exists an invertible matrix $S$ such that $S^{-1}AS$ is diagonal
\end{enumerate}

\mprop{}{Let $\lambda_1, ..., \lambda_k$ be distinct eignvalues of $\alpha$ and $v_1, ..., v_k$ be their corresponding eigenvectors. Then $\{v_1, ..., v_k\}$ is linearly independent}
\pf{}{\textbf{NEED TO DO}}

\cor{}{Let $\lambda_1, ..., \lambda_k$ be distinct eignvalues of $\alpha$. Then $E_{\lambda_1}+...+E_{\lambda_k}$ is a direct sum}
\pf{}{Assume that $w_i \in E_{\lambda_i}$ such that $\sum w_i = 0$. Without loss of generality, assume that $w_1, ..., w_L \neq 0$ and $w_{L+1} = ... = w_k = 0$. Then $w_i$ is an eigenvector of $\alpha$ with respect to $\lambda_i$. But $\sum w_i = 0$ which implies that the set $\{w_1, ..., w_L\}$ is linearly dependent. This is a contradiction to the proposition above.}

Let $\alpha \in End(V)$ and $\{\lambda_1, ..., \lambda_k\}$ all distince eigenvalues. Then the following are equivilant:
\begin{enumerate}
    \item $\alpha$ is diagonalizable
    \item $V = \sum E_{\lambda_i}$
    \item $\sum dim(E_{\lambda_i}) = n = dim(V)$
\end{enumerate}
\pf{}{\textbf{NEED TO DO}}

\dfn{Geometric and Algebraic Multiplicity}{Let $\alpha \in End(V)$ and $\lambda \in spec(\alpha)$ and $p(t)$ be the charectoristic polynomial of $\alpha$. Then the algebraic multiplicity of $\lambda$ is the largest $m_\lambda$ such that $p(t) = (t-\lambda)^m_\lambda g(t)$ and the geometric multiplicity  of $\lambda$ is $dim(E_\lambda) = d_\lambda$.}

\mprop{}{$d_\lambda \leq m_\lambda$ for all $\lambda \in spec(\alpha)$. }
\pf{}{Let $\alpha \in End(V)$ and $\{v_1, ..., v_d\}$ be a basis for $E_\lambda$. Extend it to a basis $B = \{v_1, ..., v_d, v{d+1}, ..., v_n\}$ of $V$. Then $[\alpha]_B = A$ and $p(t) = det(tI-A) = (t-\lambda)^dh(t)$. This implies that $d\leq m$.}


\thm{}{Let $\alpha \in End(V)$
\begin{enumerate}
    \item $\alpha$ is diagnalizable
    \item $p_\alpha(t)$ splits and $d_\lambda = m_\lambda$ for all $\lambda \in spec(\alpha)$
\end{enumerate}
}
\pf{}{To prove that $(2) \implies (1)$, let $\lambda_1, ..., \lambda_k$ be distinct eigenvalues. Since $p_\alpha$ splits, $p_\alpha(t) = (t-\lambda_1)^{m_{\lambda_1}}\cdot ... \cdot (t-\lambda_k)^{m_{\lambda_k}}$. This implies that $\sum m_{\lambda_i} = \sum d_{\lambda_i} = n$. Thus $\alpha$ is diagnalizable.}

\subsection{Simultaneous Diagonalization}
\qs{}{Given $\alpha, \beta \in End(V)$, under what conditions is there are a basis $B$ for $V$ such that $[\alpha]_B$ and $[\beta]_B$ are both diagonal? I.E. we need $B = \{v_1, ..., v_n\}$ such that $B$ is a basis for both $\alpha$ and $\beta$.}

\thm{}{Let $\alpha, \beta \in End(V)$ and $dim(V)$ be finite. Suppose $\alpha, \beta$ are diagonalizable. Then they are simultaniously diagonalizable if and only if $\alpha\beta = \beta\alpha$. }
\pf{}{\textbf{NEED TO DO}}

\mlenma{}{Let $\alpha \in End(V)$ and $V = W_1 \oplus W_2$ such that $W_1, W_2$ are $\alpha$ invariant. If $\alpha$ is diagonalizable, so are $\alpha$ restricted to $W_1$ and $\alpha$ restricted to $W_2$.}
\pf{}{Suppose $\alpha_1$ is $\alpha$ restricted to $W_1$ is not diagonal. Then there exists a $\lambda \in spec(\alpha_1)$ such that $d_\lambda(\alpha_1) < m_\lambda(\alpha_1)$. But this implies $d_\lambda(\alpha) < m_\lambda(\alpha)$ which is a contradiction. Thus, $\alpha_1$ and $\alpha_2$ are diagonal.
}
\mlenma{}{If $\alpha\beta = \beta\alpha$, then $E_\lambda(\alpha)$ is $\beta$ invariant for all $\lambda \in spec(\alpha)$.}
\pf{}{Let $v\in E_\lambda(\alpha)$. Then $(\alpha\beta)(v) = \alpha(\beta(v)) = \beta(\alpha(v)) = \beta \lambda v = \lambda \beta v$. Thus, $\beta(v) \in E_\lambda(\alpha)$.}


\section{09/19/2024}
\dfn{Cyclic Invariant Subspaces}{Let $\alpha \in End(V), v\in V, W = span\{v, \alpha(v), ..., \alpha^k(v), ...\}$ is a cyclic invariant subspace. $W$ is the smallest $\alpha$ invariant subspace containing $V$. Also $dim(W) = k$ implies that $\{v, \alpha(v), ..., \alpha^{k-1}(v)\}$ is a basis of $W$.}

\mlenma{}{If $p_\alpha$ splits, then there exists a basis $B$ such that $[\alpha]_\beta$ is triangular.}
\pf{}{\textbf{NEED TO DO}}

\dfn{Nilpotent}{Let $\alpha \in End(V)$. We say $\alpha$ is nilpotent with index $k$ if $\alpha^k = 0$  but $\alpha^{k-1}=0$. In this case, there exists a $w\in V$ such that $dim( span\{v, \alpha(v), ..., \alpha^k(v), ...\})=k$ and there exists an $\alpha$ invariant subspace $U$ such that $V = W \oplus U$.}

\end{document}